\chapter{Rules' Left-Hand Sides: Condition Elements}
\label{c:conditionelements}

The left-hand side (LHS) is the ``if'' part of a rule. It specifies
the conditions in working memory that must be true before the rule can
fire. The LHS is composed of \emph{condition elements}, each of which
can match objects of a particular class (and its subclasses, if
any). Condition elements can also test for particular attribute
values, and bind variables to those values. The variables can be used
only later in the same rule.

Condition elements (CEs) must be enclosed in parentheses.  CEs can be
positive, stating that certain conditions are true; or negative,
stating that certain conditions are false. A negative CE has a minus
sign (\verb|-|) before its opening parenthesis. The first CE on an LHS
must be positive.

RuleWorks performs an implicit conjunction (a logical AND operation)
on all the CEs on an LHS. A rule is eligible to fire when there are
objects that match all of its positive CEs, and there are no objects
that match any of its negative CEs. It is also possible to specify a
disjunction (a logical OR operation) of CEs.

\section{Matching the Object Class}

The first item in each CE must be the name of a declared
object class. The class name is the only required item. In
the example below, the CE (start) matches the object made
in the \tt{ON-ENTRY} statement.

\begin{quote}
\begin{verbatim}
(entry-block main)
    (object-class start)
    (on-entry (make start))
    (rule say-hello
        (start)
      -->
        (write |Hello, world!|))
(end-block main)
\end{verbatim}
\end{quote}

Example~\ref{e:obclasdecl} hows some \tt{OBJECT-CLASS} declarations
from the sample configuration program, \tt{KIWI.RUL}.

\begin{exampl}[Sample \tt{OBJECT-CLASS} Declarations]
\begin{verbatim}
(object-class part
    ^part-number
    ^name
    ^price
    ^is-expanded)

(object-class option
    (inherits-from part))

(object-class hardware-option
    (inherits-from option)
    ^takes-slot
    ^in-slot
    ^is-placed)

(object-class memory
    (inherits-from hardware-option))
\end{verbatim}
\label{e:obclasdecl}
\end{exampl}

When the class name in a CE is a parent class, objects of any of its
inheriting subclasses can also match the CE.  Given the
\tt{OBJECT-CLASS} declarations in Example~\ref{e:obclasdecl},
Table~\ref{t:matchingobclass} shows which objects match each
class. For example, an object of class \tt{MEMORY} can match a CE that
specifies any of the declared class names shown, but only an object of
class \tt{PART} can match a CE that specifies the class name
\tt{PART}.

\begin{table}[h]
  \centering
  \begin{tabular}{lllll}
    \toprule
    An Object of Class & \multicolumn{4}{l}{Matches a Condition Element of  Class} \\
    \midrule
    & \tt{PART} & \tt{OPTION} & \tt{HARDWARE-OPTION} & \tt{MEMORY} \\
    \midrule
    \tt{PART}               & Yes  & No     & No       & No  \\
    \tt{OPTION}             & Yes  & Yes    & No       & No  \\
    \tt{HARDWARE-OPTION}    & Yes  & Yes    & Yes      & No  \\
    \tt{MEMORY}             & Yes  & Yes    & Yes      & Yes  \\
    \bottomrule
  \end{tabular}
  \caption{Matching Object Classes}
  \label{t:matchingobclass}
\end{table}

An object of any visible user-declared class matches a CE that
specifies the built-in top-level class \verb|$ROOT|.

Rules can refer only to classes that are visible to the entry block or
rule block that contains them. See Chapter~\ref{c:program} for
information on making object classes visible to more than one block.

Rules can refer to a parent class or to a specific subclass. In
the following example, the rule
\tt{VERIFY-CONF:APPLICATION-NEEDS-KIWOS} applies to any one
of the software applications: \tt{KiWindows}, \tt{KiwiCalc}, or
\tt{KiwiTalk}. Conversely, rule
\tt{VERIFY-CONF:KIWITALK-NEEDS-NETWORK} applies only to
\tt{KiwiTalk}, not to \tt{KiWindows} or \tt{KiwiCalc}.

\begin{quote}
\begin{verbatim}
(rule verify-conf:application-needs-kiwos
  (active-context ^name verify-conf)
  (kiwos-application ^$instance-of <applic>)
  - (kiwos)
-->
  (make error ^severity warning ^message |Missing operating system|)
  (write (crlf) |Caution: to run application:| <applic>
     (crlf) |you need to have KIWOS, but you don't have KIWOS.|
     (crlf) |Fixup: adding KIWOS to your order.|
     (crlf))
  (make kiwos))

(rule verify-conf:kiwitalk-needs-network
  (active-context ^name verify-conf)
  (kiwitalk)
  -(network-interface)
-->
  (make error ^severity warning ^message |Missing network interface|)
  (write (crlf) |Caution: you ordered KiwiTalk, but don't have|
     (crlf) |the network interface hardware to use KiwiTalk.|
     (crlf) |Fixup: adding a network interface to your order.|
     (crlf))

  (make network-interface))
\end{verbatim}
\end{quote}

\begin{note}
  Any attribute accessed in a rule must either be declared in the
  class itself or be inherited from a parent class. RuleWorks
  generates a compile-time error if this requirement is not followed.
\end{note}

\section{Writing Attribute-Value Tests}

After the required class name, a CE can contain any number of
attribute-value tests. These are the patterns that objects in working
memory must match in order for the rule to fire. Attribute-value tests
consist of an attribute name, a predicate, and a value. You can make
combinations of tests on a single attribute by using conjunctions and
disjunctions, which are similar to AND and OR operators.

The number of attribute-value tests on the LHS determines the test
specificity of a rule. Test specificity is one of the principles used
during conflict resolution (see Test Specificity).

\subsection{Attribute Names}

The attribute name in an attribute-value test must be one
of the following constructs:
\begin{itemize}
\item the name of an attribute declared in the CE's object class (or
  declared in an ancestor of that object class)
\item the name of a built-in attribute
\end{itemize}
The attribute operator (\verb|^|) must precede the attribute name.

Given the declarations in the example above, the following are valid
CEs:

\begin{quote}
\begin{verbatim}
(part ^part-number KI-9200)
(option ^part-number KI-9200)
(hardware-option ^part-number KI-9200 ^takes-slot yes)
(memory ^part-number KI-9200 ^takes-slot yes)
\end{verbatim}
\end{quote}

The following CEs are not valid because the attributes
have not been declared for the object classes:

\begin{quote}
\begin{verbatim}
(part ^takes-slot no)
(option ^is-placed yes)
\end{verbatim}
\end{quote}

The built-in attribute names are \verb|^$INSTANCE-OF| (see
Matching the Exact Class and Matching the Instance
Identifier).

\subsubsection*{Matching the Exact Class}

If you want to match, or bind a variable to, the exact subclass of
which an object is an instance, you can use the built-in, read-only
attribute
\verb|^$INSTANCE-OF|. In the following code fragment, the second CE
binds the value of the \verb|^$INSTANCE-OF| attribute to the variable
\verb|<OPTION>|. This variable is then used in the \verb|MODIFY|
action.

\begin{quote}
\begin{verbatim}
...
(box ^$id <the-box> ^card-in-slot {[=] <len> [<] 8})
(hardware-option ^$id <the-option> ^$instance-of <option>
  ^takes-slot yes ^is-placed no )
-->
(modify <the-box>
  ^card-in-slot [(<len> + 1)] <option>
  ^card-in-slot-obj-id [(<len> + 1)] <the-option>)
...
\end{verbatim}
\end{quote}
In this example, \verb|<OPTION>| might be bound to the symbol
\verb|MEMORY|.

\subsubsection{Matching the Instance Identifier}

The built-in, read-only attribute
\verb|^$ID| stores the \tt{INSTANCE-ID} value of the object. If you
want to copy, modify, or remove an object on the RHS, you usually
first bind a variable to its identifier on the LHS. Such a variable is
called a \verb|$ID| variable.
\verb|$ID| variables give you a handle on the object that matched a
CE. \verb|$ID| variables in RuleWorks are similar to element variables
in previous versions of OPS.

You can use variables bound to values of type \tt{INSTANCE-ID} as
pointers to maintain links between objects. Example~\ref{e:3-5} shows a
RuleWorks program that uses such pointers to build a doubly linked
list of objects. Note that the variables \verb|<FIRST-DATA>| and
\verb|<LAST-DATA>| are
\verb|$ID| variables, but the variable \verb|<PREVIOUS-DATA>| is
not. Variable \verb|<NEW-DATA>|, which is bound on the right-hand
side, is not a \verb|$ID| variable. Variables \verb|<PREVIOUS-DATA>|
and \verb|<NEW-DATA>| cannot be used in \tt{MODIFY} actions as
variables \verb|<FIRST-DATA>| and \verb|<LAST-DATA>| are.

\begin{exampl}[Using \ct\tt{ID} Variables as Pointers]
\begin{verbatim}
(entry-block pointer-demo)
  (object-class data
      ^previous ; pointer to preceding datum
      ^next ; pointer to following datum
      ^value) ; actual datum

  (object-class pointer
      ^max-length ; desired length of list
      ^list compound) ; pointers to all DATA objects

  (on-entry
      (watch all)
      (make pointer ^max-length 7) ; arbitrary choice
      (make data ^previous nil ^next nil)) ; start with null pointers

  (rule init-list
      ; Initialize the list by making a DATA object with  
      ; pointers to itself
      (pointer ^$id <the-pointer>)
      (data ^$id <the-data> ^previous nil ^next nil)
    -->
      (modify <the-data> ^previous <the-data> ^next <the-data> ^value 1)
      (modify <the-pointer> ^list (compound <the-data>)))

  (rule make-list
      ; Make a new datum and add it to the list
      (pointer ^$id <the-pointer> ^max-length <max>
               ^list {list [<] <max>} ; length of list is less than <MAX>
               ^list[1] <first-data> ; bind ID of first DATA
               ^list[$last] <last-data>)  ; bind ID of last DATA
      (data ^$id <first-data> ; test ID of first DATA
            ^previous <previous-data>) ; bind backward ptr of first item
      (data ^$id <last-data> ; test ID of last DATA
            ^value <val> ; bind value of last item
            ^next <the-data>) ; bind forward ptr of last item
    -->
      (bind <new-data> ; bind a pointer to a MAKE action
          (make data ^previous <last-data> ^next <first-data> 
                     ^value (<val> + 1)))
      (modify <first-data> ^previous <new-data>)
      ; reset pointers to new item
      (modify <last-data> ^next <new-data>)
      (modify <the-pointer> ^list (compound list
      <new-data>)))
(end-block pointer-demo)
\end{verbatim}
\label{e:3-5}
\end{exampl}

\subsubsection*{Using a Bound Variable to Select an Attribute}

You can use a bound variable to select an attribute on both the LHS
and the RHS. This is useful if a program is treating object data as a
look-up table.

RuleWorks allows a variable-selected attribute to be matched against;
the variable selecting the attribute must be bound before the match
reference is made, and the value must be the name of an attribute in
that object class (see Example~\ref{e:3-6}). RuleWorks issues a run-time
warning if the attribute binding does not exist in the specified
object class.

\begin{exampl}[Variable Selection of Attributes Variable Selection of Attributes]
\begin{verbatim}
(object-class person
    ^name
    ^age
    ^phone_number
    ^address)

(object-class seek_slot ^slot_to_find)

(make person
    ^name |Joe Bloggs|
    ^phone_number 01292-474382
    ^address |17 Anderson Crescent|)

(make person
    ^name |John Doe|
    ^phone_number 0141-887-2456
    ^address |56 Alloway Drive|)

(make seek_slot
    ^slot_to_find phone_number
    ^name Dorothy)

(rule find_arbitrary_slot_datum_given_name
    (seek_slot ^slot_to_find <slot> ^name <person_name>)
    (person ^name <person_name> ^<slot> <datum>)
  -->
    (write (crlf) |Person| <name> |slot| <slot> 
        |has value| <datum>))
\end{verbatim}
\label{e:3-6}
\end{exampl}

\section{Predicates}

The test part of an attribute-value test is a predicate that specifies
the comparison to be performed (equal to, greater than, and so on)
between the atom or atoms in the attribute and the scalar or compound
value in the CE.

Every value specified in a condition element is preceded by a
predicate, either implicitly or explicitly. Values that you specify
without a predicate are implicitly preceded by the identity predicate
(see Identity, Equality, and Similarity Predicates)

\subsubsection*{Predicates and Data Types}

Table~\ref{t:3-2} shows the match predicates of RuleWorks, with the
types of attributes and values to which they can be applied. The
values referred to in the third column of Table~\ref{t:3-2} can be constants,
variables, or other expressions.

While matching condition elements to working memory, RuleWorks
performs automatic data type coercion for ``reasonable''
attribute-value tests only. Table~\ref{t:3-2} also shows which data types are
reasonable for each match predicate. All other mixed-type comparisons
yield ``no match.'' ``No match'' is not an error condition; it merely
means that the query ``A relation B'' will fail, no matter which
relation was requested (greater than, less than, equal to, and so
on). For example, the following test, which will never match, produces
a warning message at compile time, and no message or error condition
at run time:
\begin{quote}
\begin{verbatim}
^$ID < x
\end{verbatim}
\end{quote}

\begin{table}[!h]
  \begin{tabularx}{\columnwidth}{lllX}
    \toprule
    Attribute Domain & Predicate & Value Domain & Test \\
    \midrule
    ANY & \verb|==| & ANY & Identity: Same type as and equal to  (This predicate is optional in  LHS attribute-value tests. It is required in RHS relational expressions.) \\
    ANY & \verb|<>| & ANY & Nonidentity; converse of identity \\
    ANY & \verb|=| & ANY & Equality: Identical or equivalent numbers; identical symbols except for case; identical values of all other data types \\
    ANY & \verb|-=| & ANY & Inequality; converse of equality \\
    ANY & \verb|~=| & ANY & Similarity: Equal or phonetically similar symbols; equal or approximately equal numbers; identical values of all other data types \\
    ANY & \verb|-~=| & ANY & Dissimilarity; converse of similarity \\
    NUMBER & \verb|>| & NUMBER & Greater than \\
    SYMBOL & \verb|>| & SYMBOL & Lexicographically after \\
    ANY & \verb|>=| & ANY &  Greater than or equal numbers; lexicographically after or equal symbols; identical values for all other data types \\
    NUMBER & \verb|<| & NUMBER & Less than \\
    SYMBOL & \verb|<| & SYMBOL & Lexicographically before \\
    ANY & \verb|<=| & ANY & Less than or equal numbers; lexicographically before or equal symbols; identical values for all other data types \\
    ANY & \verb|<=>| & ANY &  Same type \\
    ATOM & \verb|[+]| & COMPOUND & Containment; atom is contained within the compound \\
    ATOM & \verb|[-]| & COMPOUND & Non-containment; converse of containment \\
    COMPOUND & \verb|[+]| & ATOM & Containment; compound contains atom \\
    COMPOUND & \verb|[-]| & ATOM & Non-containment; converse of containment \\
    COMPOUND & \verb|[=]| & INTEGER & Length equal \\
    COMPOUND & \verb|[<>]| & INTEGER & Length not equal \\
    COMPOUND & \verb|[>]| & INTEGER & Length greater \\
    COMPOUND & \verb|[>=]| & INTEGER & Length greater than or equal \\
    COMPOUND & \verb|[<]| & INTEGER & Length less than \\
    COMPOUND & \verb|[<=]| & INTEGER & Length less than or equal \\
    \bottomrule
  \end{tabularx}
\begin{quote}
  * Scalar predicates are those that are valid only for scalar
  attributes. The exceptions are identity and nonidentity (== and <>),
  which are also valid for comparing a compound attribute to a
  compound value.  Compound predicates are those that are valid for
  compound attributes.
\end{quote}
\caption{RuleWorks Match Predicates*}
\label{t:3-2}
\end{table}


For example, the seventh row in Table~\ref{t:3-2} states that any
\tt{NUMBER}, either \tt{FLOAT} or \tt{INTEGER}, can reasonably be
tested for being greater than any other NUMBER. Thus, assuming the
value of \verb|^LIST-PRICE| is the \verb|FLOAT| 29.95, the following
test returns \verb|TRUE|:
\begin{quote}
\begin{verbatim}
^list-price > 20
\end{verbatim}
\end{quote}
The same row also states that the greater-than predicate can be
applied to two symbols. For example, the following test is valid, and
the symbols are compared according to the lexicographic collating
sequence for the target platform (the operating system and hardware
where the executable image will run). Assume that \verb|^ANIMAL| is
bound to the symbol \tt{AARDVARK}, and the following test produces a
match:
\begin{quote}
\begin{verbatim}
^animal < zebra
\end{verbatim}
\end{quote}

A mixed type comparison such as the following generates a compile-time
warning and fails to match at run time:
\begin{quote}
\begin{verbatim}
^$ID < 42
\end{verbatim}
\end{quote}

\subsubsection*{Identity, Equality, and Similarity Predicates}

The identity predicate tests that values are of the same data type and
also equivalent to each other. It executes faster than the equality
predicate, which ignores data type and tests only for equivalent
value. Table~\ref{t:3-3} compares results of using the identity, equality, and
similarity predicates on a \tt{FLOAT} of unequal value, a \tt{FLOAT}
of equal value, and an \tt{INTEGER} of equal value.

\begin{table}[h]
  \centering
  \begin{tabular}{lllll}
    \toprule
    & & \multicolumn{3}{l}{Match if \ct\tt{X} Attribute is \ldots} \\
    \midrule
    Predicate & Example & 11.9 & 12.0 & 12 \\
    \midrule
    Identity & \verb|^x 12| & No & No & Yes \\
    Equality & \verb|^x = 12| & No & Yes & Yes \\
    Similarity & \verb|^x ~= 12| & Yes & Yes & Yes \\
    \bottomrule
  \end{tabular}
  \caption{Identity, Equality, and Similarity Predicates}
  \label{t:3-3}
\end{table}

The identity and nonidentity (\verb|<>|) predicates are the only ones
that can be applied when the attribute and the value are either both
scalar or both compound.

The identity and length-equal predicates are the only ones that can
precede the first occurrence of a variable, because they can be used
to bind a variable to the value of an attribute or to the length of a
compound attribute. The first appearance of a variable on an LHS is
where it is bound to a value. The identity predicate is the only one
that can precede a disjunction of values (see Disjunctions of Values).

\subsection{Specifying Values}

Chapter~\ref{c:workingmem} covers the general rules for specifying
values in RuleWorks. This section explains the restrictions on
constants, variables, and function calls in value expressions on the
LHS.

\subsubsection{Constants}

Constants can be symbols, integers, or floating-point numbers. There
are also two special constants: the instance identifier \verb|#0|,
which never refers to an actual WMO; and the opaque value \verb|%x0|,
which corresponds to the
null pointer provided by most other programming languages.

You cannot use any other \tt{INSTANCE-ID} or \tt{OPAQUE} constant in
source code; use them only in the command interpreter.

\subsubsection{Variables}

The first time a variable appears on an LHS, the variable is bound to
the attribute value in the object that matches the CE. All subsequent
occurrences of that variable in that rule represent the same
value. For example, suppose the LHS of a rule contains the following
CEs:
\begin{quote}
\begin{verbatim}
(memory ^price <price>) ; variable <price> is bound here
(disk ^price > <price>) ; variable <price> is tested here
\end{verbatim}
\end{quote}

If the run-time system finds a match for the first CE, the system
binds the variable \verb|<PRICE>| to the value stored in the
\verb|^PRICE| attribute of the object of class \tt{MEMORY} that
matches the CE. Suppose the variable \verb|<PRICE>| is bound to atom
129.95. Then the second CE is matched by an object of class \tt{DISK}
whose \verb|^PRICE| attribute has a value greater than 129.95.

\subsubsection{Function Calls}

You can use certain built-in or user-defined functions to specify a
value in an attribute-value test. Functions used on the LHS must not
change any objects in working memory and must always return the same
result when called with the same arguments.

The built-in RuleWorks functions that can be used on the LHS are
listed in Table~\ref{t:3-4} and Table~\ref{t:3-5}.

\begin{table}[h]
  \centering
  \begin{tabular}{ll}
    \toprule
    Name & Description \\
    \midrule
    \tt{FLOAT} &  Converts a numeric value into afloating-point number. \\
    \tt{INTEGER} &  Converts a numeric value into an integer. \\
    \tt{SYMBOL} & Converts any atom into a symbol. \\
    \bottomrule
  \end{tabular}
  \caption{LHS Functions for Scalar Values}
  \label{t:3-4}
\end{table}

\begin{table}[h]
  \centering
  \begin{tabular}{ll}
    \toprule
    Name & Description \\
    \midrule
    \tt{LENGTH} & Returns the number of elements in a compound value. \\
    \tt{NTH} & Returns the value of a specified element in a compound value. \\
    \tt{POSITION} & Finds the first occurrence of an element in a compound value. \\
    \bottomrule
  \end{tabular}
  \caption{LHS Functions for Compound Values}
  \label{t:3-5}
\end{table}

You can use the functions shown in Table~\ref{t:3-5} to make sure an
attribute-value test does not fail to match because the values are of
different types. For example, assume there is an object of class
\tt{BOX} whose \verb|^CARD-IN-SLOT| attribute is three elements
long. Assuming the variable \verb|<LIMIT>| is bound to 5.5, the
following CE does not match this \tt{BOX} object because 5.5 is a
floating-point number and the length predicates require integer
operands:
\begin{quote}
\begin{verbatim}
(box ^card-in-slot [<=] <limit>)
\end{verbatim}
\end{quote}
  However, the next CE does match:
\begin{quote}
\begin{verbatim}
(box ^card-in-slot [<=] (integer <limit>))
\end{verbatim}
\end{quote}
Function calls can include variables, but the variables
  must be bound to values. That is, the variables must
  have been used in a previous CE or previously in the
  same CE.

\subsubsection{The Quote Operator}

  In attribute-value tests, you can quote values so that
  they are not evaluated. This allows you to use any
  symbol, operator, or variable as a constant atom.

  To quote a value, precede it with the quote operator
  (\verb|//|). Using this operator is similar to enclosing an
  atom in vertical bars. For example:
\begin{quote}
\begin{verbatim}
(memory ^price // <price>)
\end{verbatim}
\end{quote}
The above CE matches an object whose class is \tt{MEMORY} and whose
\verb|^PRICE| attribute contains the symbol \verb|<PRICE>|. If the
value is quoted with vertical bars, the symbol is \verb|<price>| in
lowercase letters. If the quote operator is not there, the CE matches
an object whose \verb|^PRICE| attribute contains the value bound to
the variable \verb|<PRICE>| (or, if \verb|<PRICE>| is unbound, binds
the variable to the value of the attribute).

\subsection{Accessing Compound Attributes}

You can test, match, and bind individual elements within a compound
attribute, or the entire compound attribute.  The attribute name
without brackets indicates the entire compound value.

\subsubsection{A Single Element}

To test a single element of a compound attribute, use brackets
(\verb|[]|) and an index expression after the attribute name. This is
called element-wise access into the compound attribute. The index
expression must evaluate to an integer greater than zero. You cannot
use unbound variables in the index into a compound attribute.

The following example binds the variable \verb|<THIRD>| to the third
element of the compound attribute \verb|^CARD-IN-SLOT|:
\begin{quote}
\begin{verbatim}
(box ^card-in-slot[3] <third>)
\end{verbatim}
\end{quote}

If \verb|<THIRD>| is already bound to the symbol \tt{KEYBOARD}, the
above example matches the object shown in Figure 2-6.

Caution: When you use element-wise access into a compound attribute,
it is possible for the binding instance of a variable to cause the
match to fail.  RuleWorks does not bind \tt{NIL} or the declared
\tt{FILL} value (if any) when the index specified for an element-wise
access points past the end of the compound attribute.

In the above example, the match fails when \verb|^CARD-IN-SLOT| has
fewer than three elements, even if the variable \verb|<THIRD>| is
unbound.

\subsubsection{The Last Element}

  The special symbol \verb|$LAST| represents the index of the
  last element. \verb|$LAST| cannot be used as part of an
  expression. It must stand alone. The following example
  binds or tests the last element:
\begin{quote}
\begin{verbatim}
(box ^card-in-slot[$last] <var>)
\end{verbatim}
\end{quote}

Using \verb|$LAST| for the index expression is an element-wise
access. Therefore, the above CE never matches an empty
\verb|^CARD-IN-SLOT| attribute.

\subsubsection{The Entire Compound}

  To specify an entire compound attribute, use the
  attribute name with no index notation. This is called
  group-wise access.

  Note: When you use group-wise access of a compound
  attribute, the binding instance of a variable cannot
  cause the match to fail. If the attribute is empty, the
  variable is bound to the empty compound value
  (\verb|(COMPOUND)|).

  In the following CE, the variable \verb|<CARDS>| is bound to
  the entire compound attribute \verb|^CARD-IN-SLOT|:
\begin{quote}
\begin{verbatim}
(box ^card-in-slot <cards>)
\end{verbatim}
\end{quote}

If \verb|<CARDS>| is already bound to a compound value, the CE above
tests \verb|^CARD-IN-SLOT| and \verb|<CARDS>| for identity on an
element-by-element basis.

You must use a compound value for group-wise access of a compound
attribute. You cannot bind or compare an entire compound attribute to
a scalar value.

\subsubsection{Predicates for Compound Attributes}

RuleWorks provides a number of predicates that operate on compound
attributes only. These compound predicates allow various aspects of
the entire set of values in a compound attribute to be tested within a
condition element.

The compound predicates start and end with brackets (\verb|[]|). The
characters within the brackets specify which test (greater than, less
than, and so on) is used. Table~\ref{t:3-2} shows which predicates can
be applied to compound values.

\subsubsection{Counting the Number of Elements}

RuleWorks provides six compound predicates for testing the number of
values in a compound attribute: \verb|[=]|, \verb|[<>]|, \verb|[>]|,
\verb|[>=]|, \verb|[<]|, and \verb|[<=]|.  An example attribute-value
test:
\begin{quote}
\begin{verbatim}
^card-in-slot [>] 2 ; this test succeeds if the number of
                    ; values in ^card-in-slot is greater than 2
\end{verbatim}
\end{quote}
                  
The length-equal-to compound predicate (\verb|[=]|) allows the exact
number of elements in the compound attribute to be either bound or
compared for identity.

Just as for the identity predicate, if the expression following the
length-equal-to predicate is an unbound variable, then that variable
is bound to the actual number of elements.

If the expression following the length-equal-to predicate is a bound
variable, a constant, or some other kind of expression, then the
element count is tested for identity with the resulting value. The
value must be an integer equal to or greater than zero.

The following CE shows both uses of the length-equal-to predicate:
\begin{quote}
\begin{verbatim}
(box ^card-in-slot [=] <len> ; bind the number of values in
                             ; ^card-in-slot to the variable <len>

^card-in-slot-obj-id [=] <len>) ; match only if the number of values
                                ; in ^card-in-slot-obj-id is the same
                                ; as in ^card-in-slot
\end{verbatim}
\end{quote}

\subsubsection*{Testing for Emptiness}

Testing a compound value for emptiness is a common operation for
stacks and queues. In RuleWorks this can be done by comparing the
length of the compound attribute to zero. The following CE matches if
the compound attribute \verb|^CARD-IN-SLOT| is empty:
\begin{quote}
\begin{verbatim}
(box ^card-in-slot [=] 0)
\end{verbatim}
\end{quote}
   
\subsubsection*{Searching for a Value}

Two predicates, containment (\verb|[+]|) and non-containment
(\verb|[-]|), allow you to determine whether a compound attribute
contains a specified element value. The following CE matches if the
compound attribute \verb|^CARD-IN-SLOT| contains the value
\tt{MEMORY}:
\begin{quote}
\begin{verbatim}
(box ^card-in-slot [+] memory) ; match if memory found
\end{verbatim}
\end{quote}

The next CE uses the opposite test from the previous one:

\begin{quote}
\begin{verbatim}
(box ^card-in-slot [-] memory) ; match if memory is  NOT found
\end{verbatim}
\end{quote}
   
The containment predicates, unlike the other compound predicates, can
also be used to compare a scalar attribute value and a compound
value. For example:
\begin{quote}
\begin{verbatim}
(box ^card-in-slot-obj-id <cards>)
(memory ^$ID [+] <cards>)
\end{verbatim}
\end{quote}
In the example above, the CEs match when the identifier of the
\tt{MEMORY} object is contained within the compound value of the
\tt{BOX} object's \verb|^CARD-IN-SLOT-OBJ-ID| attribute.

By default, the containment predicates test for identity of the
compound element and the value. You can specify a different test by
placing a scalar predicate next to the containment predicate. The
compound value is then searched for an element that satisfies the
specified test and scalar value. Consider the following CE:
\begin{quote}
\begin{verbatim}
(box ^power-consumed-per-slot [-] > 20)
\end{verbatim}
\end{quote}

This could be pronounced as ``There is a \verb|BOX| object whose
\verb|^POWER-CONSUMED-PER-SLOT| attribute contains no value greater
than 20.''

The scalar predicate must be next to the scalar attribute or value.

\subsubsection*{Functions for Compound Values}

You can use three built-in RuleWorks functions, \tt{LENGTH}, \tt{NTH},
and \tt{POSITION}, with compound values on the LHS. You must pass a
bound compound variable as the first argument to each of these
functions. For example, the following CE matches a compound attribute
\verb|^CARD-IN-SLOT| that has two consecutive elements equal to
\tt{MEMORY}:
\begin{quote}
\begin{verbatim}
(box ^card-in-slot <cards>
     ^card-in-slot [((position <cards> memory) + 1)] = memory)
\end{verbatim}
\end{quote}   
The \tt{POSITION} function above finds only the first occurrence of
\tt{MEMORY} in \verb|<CARDS>|, so a rule that uses this CE fires only
once even on a \tt{BOX} object that has three or more consecutive
\tt{MEMORY} elements.

It is more efficient to use the length predicates or an index into the
compound attribute rather than the \tt{LENGTH} or \tt{NTH} functions. For
example, the following condition elements are equivalent:
\begin{quote}
\begin{verbatim}
(box ^card-in-slot <cards>
     ^card-in-slot-obj-id [=] (length <cards>))

(box ^card-in-slot [=] <len>
     ^card-in-slot-obj-id [=] <len>)
\end{verbatim}
\end{quote}
The second CE, using the length-equal predicate to bind the length of
the first compound attribute, is more efficient than the first CE.

  \section{Conjunctions of Tests}

  A conjunction is a series of tests that must all be true
  of a single attribute in an object in order for the
  match to succeed. A conjunction is similar to a logical
  AND. You specify a conjunction by enclosing the tests in
  braces (\verb|{}|). The following CE matches when there is an
  object of class \tt{HARDWARE-OPTION} whose \verb|^PRICE| attribute
  has a value between 100.00 and 500.00:
\begin{quote}
\begin{verbatim}
(hardware-option ^price { > 100.00 < 500.00 })
\end{verbatim}
\end{quote}
If you want to bind as well as test the value of an
  attribute, you can use a conjunction. For example:
\begin{quote}
\begin{verbatim}
(hardware-option ^in-slot { <slot-num> <> nil})
\end{verbatim}
\end{quote}
is equivalent to:
\begin{quote}
\begin{verbatim}
(hardware-option ^in-slot <slot-num> ^in-slot <> nil)
\end{verbatim}
\end{quote}

  If there is an object whose class name is
  \tt{HARDWARE-OPTION} or any subclass of \tt{HARDWARE-OPTION} and
  whose \verb|^IN-SLOT| attribute has a value not equal to \tt{NIL}, a
  match results. The run-time system then binds the
  variable \verb|<SLOT-NUM>| to that value.

  Conjunctions of compound predicates can also be applied
  to compound attributes. For example:
\begin{quote}
\begin{verbatim}
^card-in-slot {<cards> [+] memory [>] 2 [=] <num-cards> }
\end{verbatim}
\end{quote}
  Assuming both variables were unbound, the above
  conjunction first binds the entire compound value to the
  variable \tt{<CARDS>}. The next two tests produce a match if
  \verb|^CARD-IN-SLOT| contains at least one \tt{MEMORY} and has more
  than two elements. Finally, the above binds the number
  of elements in \verb|^CARD-IN-SLOT| to the variable
  \tt{<NUM-CARDS>}.

\section{Disjunctions of Values}

  A disjunction is a list of values any one of which can
  match the value of an attribute in an object in order
  for the match to succeed. A disjunction is similar to a
  logical inclusive OR. You specify a disjunction by
  enclosing the list of values between double angle
  brackets (\verb|<< >>|). You must not specify any predicate
  with a disjunction of values: the implicit identity
  predicate is the only valid predicate for a disjunction.

  The following CE contains a disjunction:
\begin{quote}
\begin{verbatim}
(hardware-option ^takes-slot << no false nil >> )
\end{verbatim}
\end{quote}

  Note that you must put at least one white space
  character around each side of the double angle brackets.

  Variables or any other value expressions you specify in
  a disjunction are evaluated. Consider the following
  disjunction:
\begin{quote}
\begin{verbatim}
^number << 45 <number> >>
\end{verbatim}
\end{quote}

  This matches if the \verb|^NUMBER| attribute is 45 or it is
  identical to the binding of \verb|<NUMBER>|. The next example
  shows another valid test containing a function call:
\begin{quote}
\begin{verbatim}
^number << 45 <number> (length list) >>
\end{verbatim}
\end{quote}

\section{Negative Condition Elements}

Negative CEs specify that no object that matches their pattern exists
in working memory. Negative CEs can contain any attribute-value test
allowed in positive CEs. You specify a negative CE by putting a minus
sign (\verb|-|) in front of it. Consider the rule
\tt{CHOOSE-SLOTS:PLACE-NONMEMORY} from the sample configuration
program.

\begin{exampl}[Negative Condition Elements]
\begin{verbatim}
(rule choose-slots:place-nonmemory
    (active-context ^name choose-slots)
    (box ^$id <the-box> ^card-in-slot { [=] <len> [<] 8 })
    (hardware-option ^$id <the-option>
        ^takes-slot yes
        ^$instance-of <option>
        ^is-placed no)
    -(memory ^$id <the-option>)
    -(memory ^is-placed no)
  -->
    (modify <the-box>
        ^card-in-slot [(<len> + 1)] <option>
        ^card-in-slot-obj-id [(<len> + 1)]
        <the-option>)
    (modify <the-option> ^is-placed yes ^in-slot (<len> + 1)))
\end{verbatim}
\end{exampl}

The purpose of this rule is to assign slots to options that are not
memory after all the memory cards have been placed in a slot. The
first negative CE specifies that the object that matched the third
positive CE is not of class \tt{MEMORY}. The second negative CE
specifies that there exists no object of class \tt{MEMORY} with
\verb|^IS-PLACED| attribute equal to \tt{NO}. In other words, it
specifies that all objects of class \tt{MEMORY} have been placed.

\section{Disjunctions of Condition Elements}

A disjunction of condition elements is a group of condition elements
enclosed in double angle brackets (\verb|<< >>|). The entire
disjunction matches when any one of the condition elements matches.

In general, a rule that contains CE disjunctions is equivalent to a
collection of rules, each of which corresponds to a particular branch
through each disjunction. A rule with a 2-CE disjunction and a 3-CE
disjunction has $2 \times 3 = 6$ possible branches. Although you could
express exactly the same concepts with 6 rules, by using the CE
disjunctions your code becomes more compact and more easily
maintained.

The following rule:
\begin{quote}
\begin{verbatim}
(rule test-ce-disjunctions
    << (obj-class-1 ^attr-1 <x>)
       (obj-class-2 ^attr-2 <x>) >>
  -->
    (write (crlf) |Value of attribute is| <x>))
\end{verbatim}
\end{quote}
is equivalent to the following two simple rules:
\begin{quote}
\begin{verbatim}
(rule |test-ce-disjunctions;1|
    (obj-class-1 ^attr-1 <x>)
  -->
    (write (crlf) |Value of attribute is| <x>))
\end{verbatim}
\end{quote}
and:
\begin{quote}
\begin{verbatim}
(rule |test-ce-disjunctions;2|
    (obj-class-2 ^attr-2 <x>)
  -->
    (write (crlf) |Value of attribute is| <x>))
\end{verbatim}
\end{quote}
There is a restriction on the use of CE disjunctions: if a variable is
bound in one branch of a CE disjunction, and the variable appears
later on in the rule, then it must be bound in all branches of the CE
disjunction. A variable \verb|<X>| can be bound in only one branch of
the disjunction provided that \verb|<X>| does not subsequently appear
elsewhere in the rule.

It is not recommended that you do not write CE disjunctions such that
the same object can match more than one branch. Doing so causes the
rule to fire more than once on the same object, a result you probably
would not expect.

When a rule contains a disjunction of CEs (see Disjunctions of
Condition Elements) the test specificity of each instantiation is
calculated separately (see Test Specificity). The class specificity of
each instantiation is calculated separately, too.

\section{Test Specificity}

If two or more instantiations in the conflict set have equal priority
after refraction, recency, and class specificity have been applied
(see Chapter~\ref{c:intro}), RuleWorks orders the instantiations
according to their test specificity. An instantiation's test
specificity is determined by the number of attribute-value tests in
the left-hand side of the rule to which the instantiation refers.

Each of the following items adds one to the test specificity of a
rule:
\begin{itemize}    
\item An object class name. The class name counts as one even if the
  condition element is negated.
\item A disjunction of values. The entire disjunction counts as one
  test no matter how many values it contains.
\item A predicate followed by any value expression, except an unbound
  variable or within a disjunction of values. This includes the
  implied identity predicate.
\end{itemize}
   
In a conjunction of tests, each test adds one to the test specificity,
subject to the restrictions above. For example, assuming the two
variables are not bound, the following CE contains only two tests:
\begin{quote}
\begin{verbatim}
(box ^$id <the-box> ^card-in-slot { [=] <len> [<] 8 })
\end{verbatim}
\end{quote}
The two tests in this CE are the class name \tt{BOX} and the
length-equal predicate followed by the value 8. The two variables are
unbound, so their presence does not add to the test specificity.
\begin{note}
  The following items do not count towards test specificity:
  \begin{itemize}
  \item An explicit attribute name
  \item The binding instance of a variable
  \end{itemize}
\end{note}

%%% Local Variables:
%%% mode: latex
%%% TeX-master: "rwug"
%%% End:

\chapter{Compiling, Linking, and Running Programs}

You compile a RuleWorks program with the command
\tt{rulework}, which invokes the RuleWorks compiler to
process the source code and verify that it contains no
syntax errors or violations of the language semantics.
You can compile only one source file with each \tt{rulework}
command. If no errors occur, the compiler generates an
intermediate C language file.

You can also direct the compiler to produce a listing
file or an error file.

This chapter covers the following topics:
\begin{itemize}
\item Compiling RuleWorks Programs
\item Compiling Generated C Files
\item Linking RuleWorks Programs
\item Dividing a Program into Blocks
\item Running RuleWorks Programs
\end{itemize}

\section{Compiling RuleWorks Programs}

The source code for a RuleWorks program is contained in
one or more ASCII text files. Each file must contain at
least one complete block; you can put more than one block
into a file, but you must not split a block between
files. A RuleWorks program can consist of a single entry
block or multiple entry, declaration, and rule blocks
(see Dividing a Program into Blocks for details on
compiling multiple blocks).

The syntax for the RULEWORKS command is shown below:
\begin{quote}
\tt{rulework} \it{file-spec} [\it{qualifier}] \ldots
\end{quote}
\it{file-spec} names the source file to be compiled into a C
language file. The default file type for source files is
\verb|rul| if none was supplied, and there is no period (\verb|.|) in
the file spec.

The default file name for most output files is the name of the source
file, with an extension (or file type) that depends on the kind of
output file being produced. The following table summarizes RuleWorks's
file naming conventions.

\begin{tabularx}{\columnwidth}{XXX}
  \toprule
  File's Contents & Name & Extension/Type \\
  \midrule
  RuleWorks source code & --- & \tt{.rul} \\\addlinespace
  Generated C code & Same as source file & \tt{.c} \\\addlinespace
  \raggedright Compiled declaration blocks & First eight alpha-numeric characters of the characters of the declaration block's name (one file for each block) & \tt{.use} \\\addlinespace
  Error messages & Same as source file & \tt{.err} \\
  \bottomrule
\end{tabularx}

You can specify a different name by including a file-spec
with the Error or Output qualifiers. For example, the
following command generates the intermediate file
\verb|config.c| and the error file \verb|bugs.err|:

\begin{quote}
\begin{verbatim}
C:\> ruleworks config /err=bugs
\end{verbatim}
\end{quote}

The following example compiles the source file \verb|config.rul|
and generates the intermediate file \verb|my_config.c|:

\begin{quote}
\begin{verbatim}
% ruleworks -output=my_config config
\end{verbatim}
\end{quote}

\subsubsection{Qualifier}

Qualifier specifies an instruction to the compiler. You can put
qualifiers before or after the file specification(s). Table 8-3 shows
all the valid qualifiers and values (defaults are in bold print).

Table 8-2 explains how to delimit RULEWORKS command
qualifiers:

Table 8-2. How to Delimit RULEWORK Command Qualifiers

\begin{quote}
\begin{tabular}{ll}
  \toprule
  If you are running this operating system\ldots & Use this character\ldots \\
  \midrule
  Microsoft Windows & Either minus (\verb|-|) or slash (\verb|/|) \\
  OpenVMS & Slash only \\
  Compaq Tru64 UNIX or Linux & Minus only \\
  \bottomrule
\end{tabular}
\end{quote}

You can use either an equal sign (\tt{=}) or a colon (\verb|:|)
between a qualifer and its value. You can shorten qualifier and value
names to the smallest unique leading substring. Table 8-3 shows the
full names and the abbreviations.

\begin{note}
  The RuleWorks command line is common to all platforms, with the
  exception of the character used to delimit the qualifiers (see
  above). In other words, the qualifiers themselves are
  operating-system independent.  Thus, the information in this guide
  is valid for RuleWorks on any platform.
\end{note}

However, some command-line features that you may expect on your
operating system may not work with RuleWorks.  For example:

\begin{itemize}
\item you must use a delimiter before each qualifier
\item you must put a space before each qualifier as well as before the
  filename parameter
\item you must not use /NOqualifier to disable a qualifier
\end{itemize}
  

TODO
Table 8-3. RuleWorks Compiler Qualifiers

\begin{longtable}{p{5cm}p{10cm}}
  \toprule
  Syntax & Description \\
  \midrule
  \verb|/DEBUG=YES| or \verb|-d=y| or \verb|/D| or \verb|-d|
         &
           Includes debugging tables in the generated C code,
           enables the \tt{DEBUG} and    
           \tt{TRACE} actions, and
           automatically invokes the
           RuleWorks command
           interpreter. The        
           interpreter is invoked   
           immediately after the    
           \tt{ON-ENTRY} actions, if any,
           of the first entry block
           (of those contained in   
           the source file) to be 
           run.                     
           \tt{YES} is the default if you 
           specify \tt{DEBUG} without a   
           value. \\\addlinespace
  \raggedright
  \verb|/D=MAYBE| or \verb|-d=m| 
         &
           Includes debugging tables
           in the generated C code 
           and enables the \tt{DEBUG} and
           \tt{TRACE} actions, but does
           not automatically invoke
           the RuleWorks command
           interpreter. You can put
           the \tt{DEBUG} action in any
           executable statement to  
           invoke the interpreter, 
           or you can call the API
           routine, \verb|rul_debug|, from
           your system debugger. \\\addlinespace
  \raggedright
  \verb|/D=NO| or \verb|-d=n|
         &
           Does not include
           debugging tables in the 
           generated C code, does 
           not enable the \tt{DEBUG} and 
           \tt{TRACE} actions, and does 
           not invoke the RuleWorks 
           command interpreter. 
           Also, makes the 
           compilation faster and 
           reduces the size of the 
           output file. 
           This makes it impossible 
           for you to use the 
           RuleWorks debugger on any 
           entry block in the file 
           being compiled. 
           \tt{NO} is the default if you 
           do not specify \tt{DEBUG} at 
           all. \\\addlinespace
  \raggedright
  \tt{/ERRORS}[\tt=\it{file-spec}] or \tt{-e}[\tt=\it{file-spec}] 
         &
           Sends error messages to a
           file instead of 
           displaying them on your 
           screen. The default 
           filename is the same as 
           the source file, with 
           extension \tt{.err}, in the 
           current directory. \\\addlinespace 
  \raggedright
  \tt{/OPTIMIZE=REINVOCATION} or \tt{-op=r} or \tt{/OP} or  \tt{-op}
         &
           Retains matching          
           information when the      
           entry block exits. Never  
           releases portions of the  
           memory it uses, but may   
           improve performance       
           dramatically when the     
           entry block is called     
           more than once.           
           \tt{REINVOCATION} is the       
           default if you specify    
           \tt{OPTIMIZE} without a value. \\\addlinespace
  \raggedright
  \tt{/OP=SPACE} or \tt{-op=s} 
         &
           Clears memory after     
           exiting the entry block.
           Not recommended if the  
           entry block is called   
           more than once          
           \tt{SPACE} is the default if 
           you do not specify      
           \tt{OPTIMIZE} at all. \\\addlinespace
  \raggedright
  \tt{/OUTPUT=}\it{file-spec} or \tt{-ou=}\it{file-spec}
         &
           Names the generated C   
           file. The default is to 
           use the same name as the
           source file with the    
           extension \tt{.c}, in the    
           current directory. \\\addlinespace
  \tt{/QUIET} or \tt{-q}
         &
           Suppresses the RuleWorks
           copyright notice. \\\addlinespace
  \tt{/USEDIRECTORY=}\it{dir-spec} or \tt{-u=}\it{dir-spec} 
         &
           Names a directory (or  
           path) where .USE files,
           which contain compiled 
           declaration blocks, are
           located. The default is
           the current directory. \\
  \bottomrule
\end{longtable}

The following sections explain how to use each
qualifier.

Invoking the RuleWorks Command Interpreter (Debug)
Invoking the <delayed>(<pname>) Command Interpreter
(Debug)

RuleWorks includes language-specific debugging features
that provide information, such as the contents of working
memory and the conflict set, that is not accessible to
your system debugger. To use the RuleWorks debugging
features, compile your entry block(s) with the Debug
qualifier set to \tt{YES}. For example:
\begin{quote}
\begin{verbatim}
$ rulework /debug=yes config
\end{verbatim}
\end{quote}  
The RuleWorks run-time system invokes the command interpreter
immediately after the \tt{ON-ENTRY} actions (if any) of the first
entry block that runs, before the first recognize-act cycle
executes. At the interpreter prompt, you can give the debugging
commands explained in Chapter~\ref{c:debugging}.

If you want to use your system debugger but not the RuleWorks command
interpreter, use the Debug qualifier set to \tt{MAYBE}. This causes
the generated C file to include the same information as for \tt{YES},
but does not invoke the RuleWorks command interpreter
automatically. You can invoke it from your system debugger by calling
the API routine \verb|rul_debug|, or by putting the \tt{DEBUG} action
in your source code.

After you've finished debugging your entry blocks, you can increase
the speed of compilation and decrease the size of the generated C
files by compiling without the Debug qualifier (or with Debug set to
\tt{NO}).

\subsubsection{Saving Error Messages (Errors)}

By default, messages from the RuleWorks compiler appear
on your terminal only. You save them in a file instead by
using the Errors qualifier when you compile. For example:

\begin{quote}
\begin{verbatim}
C:\> rulework config /err=bugs
\end{verbatim}
\end{quote}  

If you specify the Errors qualifier with no value, the default is the
source file name with the \tt{.err} file type, in your current
directory.

The Errors qualifier affects compile-time messages only.
Use the \tt{DEFAULT} command to redirect run-time messages.

\subsubsection{Producing a Listing File (List)}

A listing file is useful for debugging because it provides information
about errors the compiler detects during compilation together with the
source code listing.  In interactive mode, the compiler produces a
listing file only if you specify the \tt{/LIST} qualifier. The default
file specification for the listing file consists of the name of the
RuleWorks source file with a \tt{.lis} file type. For example, the
following command causes the compiler to produce the listing file
\tt{config.lis}:

\begin{quote}
\begin{verbatim}
System> rulework config /list
\end{verbatim}
\end{quote}  

If you want to give the listing file a different name, use the
\tt{/LIST} qualifier with a file specification. For example, to
compile the program \tt{config.rul} naming the listing file
\verb|config_list.lis|, use the following command:

\begin{quote}
\begin{verbatim}
System> rulework config /list=config_list
\end{verbatim}
\end{quote}  

In batch mode, the compiler produces a listing file by default. To
suppress the listing file, use the \tt{/NOLIST} qualifier.

\subsubsection{Controlling the Case of C Function Names (Names)}

All RuleWorks block names and external routine names become C function
names and are visible to your linker.  Because of the case-sensitivity
of most C compilers, you may need to specify whether RuleWorks
generates C function names in uppercase or lowercase.

The default is uppercase function names. If you need lowercase, use
the Names qualifier. For example:

\begin{quote}
\begin{verbatim}
% <lcname> config -n
\end{verbatim}
\end{quote}

\subsubsection{Optimizing RuleWorks (Optimize)}

While a RuleWorks program is running, the match phase of the
recognize-act cycle consumes the most CPU time. To reduce time spent
matching rules to WMOs, RuleWorks uses extra memory to save partial
instantiation and conflict set information between cycles. This allows
the run-time system merely to update match information every cycle,
rather than recreate it from scratch. (RuleWorks uses a variant of the
RETE match algorithm; see Rule-based Programming with OPS5 for more
information on writing rules efficiently.)

When an entry block exits, the memory used for its match information
is normally freed. This means, however, that if the entry block is
called again, then all match information must be recreated. This may
have a serious impact on performance if you call the same entry block
many times. Use the Optimize qualifier with the \tt{REINVOCATION}
value to keep match information in memory between calls to an entry
block.

For example, assuming the entry block \tt{VERIFY} is called
repeatedly, the following command retains some memory but
may greatly reduce execution time:

\begin{quote}
\begin{verbatim}
% rulework -opt=r verify
\end{verbatim}
\end{quote}

When deciding whether to use \tt{REINVOCATION}, the determining factor
is how much working memory, of classes visible to the called entry
block, changes between the time that entry block returns and the time
it is called again. If more WMOs stay the same than are created,
changed, or deleted, then use \tt{REINVOCATION}. Conversely, if more WMOs
are created, changed, or deleted than are left the same, then you
should compile the called entry block with Optimize left at the
default, \tt{SPACE}.

\begin{note}
  Language semantics are not affected by this qualifier, only the cost
  of entry block initialization at run-time and maximum memory usage.
\end{note}

\subsubsection{Naming the C File (Output)}

To produce a C file with a different name from your source file, use
the Ouput qualifier with a new file name. The following command
compiles the source file \tt{verify.rul} into the generated file
\tt{config.c}:

\begin{quote}
\begin{verbatim}
C:\> rulework /out=config verify
\end{verbatim}
\end{quote}  

The default for output files is the same name as the
source file, with the \tt{.c} file type, in the current
directory.

\subsubsection{Suppressing the Copyright Notice (Quiet)}

The RuleWorks compiler usually displays several lines of copyright and
version each time it starts. To turn off this display, use the Quiet
qualifier. For example:

\begin{quote}
\begin{verbatim}
% rulework config -q
\end{verbatim}
\end{quote}

\subsubsection{Storing Declarations Separately (Usedirectory)}

If you have declaration blocks that are shared by many entry blocks,
you may find it convenient to keep their compiled declarations files
(\tt{.use}, not \tt{.c}) in a separate directory, perhaps one that is
accessible to other people. To do this, first place the \tt{.use}
files generated by compiling the declaration blocks in one directory.
From then on, compile all files that contain either those declaration
blocks, or entry blocks and rule blocks that use those declaration
blocks, with the Usedirectory qualifier set to the appropriate
location. For example:

\begin{quote}
\begin{verbatim}
C:\rules\work> rulework config -usedir=d:\rulework\project\decls
\end{verbatim}
\end{quote}

By default, \tt{.use} files are created in and read from the current
directory (in this example, \verb|C:\rules\work|). The Usedirectory
qualifier resets the directory for \tt{.use} files (in this example,
to \verb|D:\rulework\project\decls|).

\section{Compiling Generated C Files}

RuleWorks runs on several hardware/operating system pairs, and
supports several C compilers. The RuleWorks Release Notes list the
supported C compilers and platforms. On Digital UNIX systems, you
compile C files generated by RuleWorks just as you would your own C
files. (See your C language documentation for details.)

On non-Digital UNIX systems, you must tell your C compiler where the
RuleWorks include files are located (this applies both to generated
files and to your own C files that call RuleWorks API routines). The
syntax for this is shown below:

\begin{quote}
\begin{verbatim}
$ cc file-spec... /include_directory=rul$library:
C:\> cc file-spec... -Ic:\rulework
\end{verbatim}
\end{quote}

On OpenVMS systems, you must also compile the generated C files using
the default floating-point arithmetic. Table 8-4 shows which C command
qualifiers are restricted.

TODO
Table 4-4. VAX C and DEC C Compiler Qualifiers

+-------------------------------------------------------+
| Platform      | Compiler | Restrictions               |
|---------------+----------+----------------------------|
| OpenVMS VAX   | VAX C    | You must not use \verb|/g_float|  |
|---------------+----------+----------------------------|
| OpenVMS VAX   | DEC C    | You must not use           |
|               |          | \verb|/float=g_float|             |
|---------------+----------+----------------------------|
| OpenVMS Alpha | DEC C    | You must not use           |
|               |          | \verb|/float=d_float|             |
+-------------------------------------------------------+


\section{Linking RuleWorks Programs}

You link compiled C files generated by RuleWorks the same as your own
C files, but with the addition of the RuleWorks run-time
library. (Check your C language and/or linker documentation for
information on linking with your system debugger.) This section covers
the following topics:

\begin{itemize}
\item Linking multiple modules with the RuleWorks run-time system
\item Linking external routines
\item Linker errors
\end{itemize}

\subsubsection{Linking with the RuleWorks Run-Time System}

When you have successfully compiled all the modules of your program,
you need to link them and the RuleWorks run-time system together to
produce an executable file.  This creates a program that can run on
systems that do not have the RuleWorks run-time system installed. You
do this with your linker's library option.

For example, the following commands link the modules \tt{phone.obj} and
\tt{phonbook.obj} with the RuleWorks object library.

Using DEC C on OpenVMS:

\begin{quote}
\begin{verbatim}
$ link phone.obj,phonbook.obj,rul$library:rul_rtl.olb/library
\end{verbatim}
\end{quote}  

Using DEC C on Digital UNIX:

\begin{quote}
\begin{verbatim}
% c89 phone.o phonbook.o -lrulrtl
\end{verbatim}
\end{quote}

Using WATCOM on MS-DOS:

\begin{quote}
\begin{verbatim}
C:\> wcl386 phone.obj phonbook.obj \rulework\rul_rtlw.lib
\end{verbatim}
\end{quote}  

There are three versions of the run-time library for Microsoft
Windows: \verb|RUL_RTLW.LIB|, \verb|RUL_RTLM.LIB|, and
\verb|RUL_RTLB.LIB|. Use \verb|RUL_RTLM.LIB| with a Microsoft C
compiler, \verb|RUL_RTLB| with the Borland C compiler, and substitute
the appropriate command for WCL386 in the example above.

\subsubsection{Linking External Routines}

The procedure for compiling an external routine depends, of course, on
the programming language in which the routine is written. See the
appropriate language user's guide for instructions on how to compile
an external routine; see Compiling RuleWorks Programs for instructions
on how to compile RuleWorks modules.

Compiling an external routine creates an object file
whose name you can include in the command you use to link
your RuleWorks modules to produce an executable image.
For example, if the compiled RuleWorks modules are
\tt{stockinit.obj} and \tt{dostock.obj}, and the external routine
object file is \tt{stocksub.obj}, the link command on OpenVMS
is:

\begin{quote}
\begin{verbatim}
$ link/exe=stock stockinit,dostock,stocksub,rul$library:rul_rtl/library
\end{verbatim}
\end{quote}  

This command links the RuleWorks object files with the object file
\verb|stocksub.obj| created by another compiler and the RuleWorks
run-time system. You can use your system debugger to debug external
routines.

\subsubsection{Effect of Declaration Blocks on Linking}

Every time you compile a declaration block, the generated code (and
the \tt{.use} compiled declarations file) includes a special generated
time/date/name linkage point (that is, a global variable). When other
blocks use that declaration block they include a reference to the
specific variable named in the compiled declaration file.

These variable names are of the format:

\begin{quote}
\begin{verbatim}
RUL_decl-block_date-compiled_time-compiled
\end{verbatim}
\end{quote}

For example:

\begin{quote}
\begin{verbatim}
RUL_KB__OCT_27_93_160707
\end{verbatim}
\end{quote}  

The reason we do this is to guarantee that in a final image all the
code that was generated based upon some set of declarations, was also
linked based upon the exact same version of those declarations. If
your build procedure fails to ensure this, you get unresolved linker
references to symbols similar to those shown above. When you get such
linker warnings, recompile each module that uses the specified
declaration block. (Remember to specify the correct directory - see
Storing Declarations Separately (Usedirectory))

If you are using project build utility, you can automate this checking
by adding a dependency for each of your RuleWorks-generated \tt{.c}
files. These dependencies should state that the \tt{.c} file depends
not only on the \tt{.rul} file, but also on all the shared \tt{.use}
files. See the section of this chapter, Sample Make File for a sample
Wmake file.

\section{Dividing a Program into Blocks}

Many programs are too large and complex to be conveniently contained
in one source file. In general, when a block is a non-trivial size,
Digital recommends that you put each block into a separate file. You
can put more than one block into one file, but you cannot split one
block into more than one file. If a single block becomes too large to
manage reasonably in a single file, you can divide it into multiple
blocks of the appropriate types, each in its own file. See Chapter 5
for details on using RuleWorks block constructs.

There are two advantages to using multiple files: first, your
development cycle (edit, compile, link, debug) can be faster because
you may be able to edit and compile only one small file instead of the
entire large program; and second, you can apply compiler qualifiers,
especially Debug and Optimize, selectively to specific blocks.

Even if all your RuleWorks code fits into one file, you may want to
split it into modular subsystems, each of which performs a specific
task, or group of tasks, within the program. You can create subsystems
as follows:

\begin{itemize}
\item one or more declaration blocks
\item one or more entry blocks
\end{itemize}

By separating class declarations into more than one declaration block,
and choosing the arguments to the entry blocks carefully, you can
select an appropriate information bandwidth between your
subsystems. (See Chapter~\ref{c:program} for more details on private
and shared data in RuleWorks.)

\begin{note}
  If a declaration block is used by entry or rule blocks that are
  contained in more than one file, Digital recommends the following:
  \begin{itemize}
  \item put the declaration block in a file by itself
  \item make the filename the same as the declaration block name.
  \end{itemize}
\end{note}

This makes debugging much easier. You may find it easiest simply to
put every block in its own file.

Follow these steps when compiling multiple files:

\begin{enumerate}
\item RuleWorks compile all files that contain declaration blocks
  first.

  If an entry or rule block uses a declaration block, you must compile
  the declaration block before you compile the entry or rule
  block. Inside an entry block or rule block, declarations must come
  before any executable statements.

  Compiling a declaration block creates a file whose name is the first
  eight characters of the block name and whose type is \verb|.use|
  (for ``used'' declarations). The compiled file stores the following
  information:

  \begin{itemize}
  \item Object class names and inheritance structure
  \item Attribute names and characteristics 
  \item External routine names and parameter types
  \end{itemize}
  
\item RuleWorks compile files that do not contain any declaration
  blocks second.

  You can compile entry and rule blocks in any order, even if the
  entry blocks call each other and activate the rule blocks. You must
  compile all the blocks before you link.

\item C compile all the generated files.

  (Check the section titled Compiling Generated C Files for
  restrictions.)

\item Link all the object files.

  (See the section of this chapter titled Linking RuleWorks Programs
  for specific instructions.)

\item Run the executable, debug as needed, and repeat the procedure.

  If you edit a declaration block, remember to recompile all the files
  that contain entry and rule blocks that use it, as well as the file
  that contains the declaration block itself.

  If you edit an entry block, you should not need to recompile any
  file except the one that contains the entry block.
\end{enumerate}

The following example shows these steps applied to a sample program
called \tt{phone}. This program consists of a main routine and two
subroutines in C, three RuleWorks entry blocks, and one declaration
block. Figure 8-1 illustrates the files used in the command sequence
shown in the example, Modular Compilation.

\begin{quote}
\begin{verbatim}
$ rulework phonbook.rul (1)
$ rulework phon_reg.rul (2)
$ rulework phonlook.rul
$ cc /incl=rul$library phone.c
$ cc /incl=rul$library phonbook.c (3)
$ cc /incl=rul$library phon_reg.c
$ cc /incl=rul$library phonlook.c
$ link phone,phonbook,phon_reg,phonlook,rul$library:rul_rtl/lib (4)
$ run phone (5)
...
\end{verbatim}

changes to entry block \verb|LOOKUP_PHONE_NUMBER| in file
\verb|phonlook.rul|

\begin{verbatim}
...
$ rulework phonlook.rul
$ cc /incl=rul$library phonlook.c
$ link phone,phonbook,phon_reg,phonlook,rul$library:rul_rtl/lib (6)
$ run phone (7)
...
\end{verbatim}
\end{quote}
Key to the table and figure Modular Compilation:
\begin{itemize}
\item[\tt{(1)}] The \verb|phonbook.rul| file contains the only
  declaration block, so it is RuleWorks compiled first.

\item[\tt{(2)}] The \verb|phon_reg.rul| and \verb|phonlook.rul| files
  contain entry blocks and private declarations, but no declaration
  blocks. They are RuleWorks compiled after \verb|phonbook.rul|.

  The file \verb|phone.c| contains all the original C language source
  code, including the main routine. Note that it is compiled with the
  \verb|RUL$LIBRARY| logical (this syntax is for OpenVMS only).

\item[\tt{(3)}] The generated files are all C compiled, again with the
  \verb|RUL$LIBRARY| location specified.

\item[\tt{(4)}] All modules are linked with the RuleWorks object
  library.

\item[\tt{(5)}] The program is run (this syntax is OpenVMS only).
\end{itemize}

If an entry block or rule block contains errors, you can edit that
block and recompile its file separately.  However, if you edit a
declaration block that is used by an entry block or rule block, you
must recompile the file that contains the entry or rule block as well
as the file that contains the declaration block.

In this example, after editing the entry block named
\verb|LOOKUP_PHONE_NUMBER|, which is contained in the file
\verb|phonlook.rul|, the only required recompilation is that of
\verb|phonlook.rul|.

If the declaration block contained in file \verb|phonbook.rul| were
edited, all the RuleWorks files would have to be recompiled. See the
section of this chapter, Effect of Declaration Blocks on Linking for
information on linker errors that can result from mismatched
\verb|.use| files.

Figure 8-1. Modular Compilation

If your program consists of many modules, you probably want to use a
project build utility such as Make (or equivalents such as NMake,
WMake, or MMS). These tools allow you to automate your system build
such that the minimum recompilations are performed. They also ensure
that all necessary recompilations are performed. Example~\ref{e:8-2}
shows a sample Wmake file for the examples shipped in the RuleWorks
kit.

\begin{exampl}[Make File]
\begin{verbatim}
# Make file for building RuleWorks example programs  using Watcom.
# Build with `wmake -f examples.mak'.
RULDIR = # Assume RuleWorks is in directory
RULECOMP = $(RULDIR) # Command to invoke RuleWorks compiler
RULRTL = $(RULDIR)_rtl.lib # Syntax for including run-time library in link
CC = wcl386 # Command to invoke C compiler
LINK = wcl386 # Command to invoke linker
CFLAGS = -c -I$(RULDIR) # compile only, .h file location
LINKFLAGS = -l=dos4g -k63000 # DOS extender, stack size
SUFFIXES: .rul

.rul.c : # Rule for RuleWorks to C (.rul to .c)
        $(RULECOMP) $<

.c.obj : # Rule for compiling C program
        $(CC) $(CFLAGS) $<

.obj.exe : # Rule for linking
        $(LINK) $(LINKFLAGS) $< $(RULRTL)

all : count.exe advent.exe tourney.exe .symbolic
\end{verbatim}
\label{e:8-2}
\end{exampl}

\section{Running RuleWorks Programs}

Once you have compiled and linked your program, run the executable
file. This procedure is the same whether your main program is a
RuleWorks entry block or a routine written in another language. For
example, on OpenVMS systems:

\begin{quote}
\begin{verbatim}
$ run phone
\end{verbatim}
\end{quote}

or on UNIX:

\begin{quote}
\begin{verbatim}
% ./phone
\end{verbatim}
\end{quote}

or on MS-DOS:
\begin{quote}
\begin{verbatim}
C:\> phone
\end{verbatim}
\end{quote}

When an entry block receives control, it invokes the RuleWorks
run-time system. Depending on how the entry block was compiled, and
whether it contains a \tt{DEBUG} action, the run-time system may
invoke the command interpreter.

TODO
Table 8-6. Run-time System

+-------------------------------------------------------+
| If the entry block |                                  |
| was compiled with  | The run-time system...           |
| the ...            |                                  |
|--------------------+----------------------------------|
|                    | executes the \tt{ON-ENTRY} actions    |
|                    | (if any) and the match and       |
|                    | select steps of the first        |
| Debug qualifier    | recognize-act cycle, and then    |
| set to \tt{YES}         | pauses the entry block and       |
|                    | invokes the command interpreter. |
|                    | Pauses the entry block again     |
|                    | after its ON-EXIT actions (if    |
|                    | any).                            |
|--------------------+----------------------------------|
| Debug qualifier    | executes all the ON-ENTRY        |
| set to \tt{MAYBE} and   | actions and the match and select |
| the \tt{ON-ENTRY}       | steps of the first recognize-act |
| contains a DEBUG   | cycle, and then pauses the entry |
| action             | block and invokes the command    |
|                    | interpreter.                     |
|--------------------+----------------------------------|
|                    | executes the ON-ENTRY actions    |
| Debug qualifier    | (if any), continues firing rules |
| set to MAYBE and a | until it finishes the act phase  |
| rule contains a    | for the rule that contains the   |
| DEBUG action       | DEBUG action, and then pauses    |
|                    | the entry block and invokes the  |
|                    | command interpreter.             |
|--------------------+----------------------------------|
|                    | executes the ON-ENTRY actions    |
| Debug qualifier    | (if any), continues to           |
| set to NO          | completion, and returns without  |
|                    | ever invoking the command        |
|                    | interpreter.                     |
+-------------------------------------------------------+

At the RuleWorks command interpreter, you run recognize-act cycles by
entering the \tt{RUN} command:

\begin{quote}
\begin{Verbatim}[commandchars=\\\{\}]
\RWP\cmd{run}
\end{Verbatim}
\end{quote}

You can control the number of recognize-act cycles the run-time system
executes by entering the \tt{RUN} command with an integer. For
example, to execute four recognize-act cycles, specify:

\begin{quote}
\begin{Verbatim}[commandchars=\\\{\}]
\RWP\cmd{run 4}
\end{Verbatim}
\end{quote}

The integer refers to the global (to the program) rule-firing counter,
not the local (to this invocation of the entry block) counter.

See Chapter~\ref{c:debugging} for information on the other commands
available at the \verb|RuleWorks>| prompt.

%%% Local Variables:
%%% mode: latex
%%% TeX-master: "rwug"
%%% End:

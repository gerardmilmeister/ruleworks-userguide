\chapter{Portability and Compatibility Issues}
\label{c:portability}

This chapter provides information on differences between platforms. It
includes:
\begin{itemize}
\item File specifications
\item Input Source and Output Destination
\item Hardware and Operating System Restrictions
\item Internationalization Issues 
\item RuleWorks and OPS5 Compatibility Issues
\end{itemize}

\section{File Specifications}

The restrictions on file names vary according to operating system. The
following characteristics are typical of file names on various
platforms, but are restricted when using RuleWorks:

Brackets (\verb|[]|) signify an index into a compound attribute. The
comment character for RuleWorks is a semicolon (\verb|;|). Therefore,
if the file specification you supply includes any of these characters,
you must enclose it in vertical bars (\verb,||,).

The following command illustrates VMS-style file specifications:
\begin{quote}
\begin{verbatim}
@ |USER$1:[SMITH.WORK]INITIALIZE-MEMORY.WM|
\end{verbatim}
\end{quote}

If your file specification needs to be passed to the operating system
in lowercase letters, you must enclose it in vertical bars (\verb,||,).

The following command illustrates UNIX-style file specifications:
\begin{qv}
@ |usr/users/smith/work/initialize-memory|
\end{qv}

If your file specification contains any of the following characters,
you must enclose the file specification in vertical bars (\verb,| |,):
caret (\verb|^|), tilde (\verb|~|), number sign (\verb|#|), percent
sign (\verb|%|), ampersand (\verb|&|), parentheses (\verb|()|), and
braces (\verb|{}|).

The following command illustrates MS-DOS-style file specifications:
\begin{qv}
@ |C:\smith\work\init%mem.com|
\end{qv}

\section{Input Source and Output Destination}

If you do not use the \tt{DEFAULT} command or action to specify
otherwise, RuleWorks uses standard input and standard output
(\verb|SYS$INPUT| and \verb|SYS$OUTPUT| on VMS systems).

\section{Hardware and Operating System Restrictions}

On VMS systems, you cannot write more than 252 atoms at a time.

\section{Internationalization Issues}

On VMS systems, RuleWorks collates symbols according to the Digital
Multinational Character Set.

This does not follow the I18N Guidelines. Please use the specified NCS
routines.

On MS-DOS, RuleWorks does not change the active code page (see your
DOS manual for the \tt{chcp} command). On Windows NT, RuleWorks does not
change the Internationalization setting.

\section{Compatibility Issues}

There are two major differences and numerous minor differences between
RuleWorks and OPS5:
\begin{itemize}
\item The first major difference is that RuleWorks makes no attempt to
  be 100\% compatible with the public domain OPS5. Constructs like
  \tt{LITERALIZE}, \tt{LITVAL}, and \tt{VECTOR-ATTRIBUTE} are not
  available in RuleWorks.

\item The second major difference is that the public domain OPS5
  provides none of the modularity features that RuleWorks provides,
  although most of the other integration features were provided.

  These modularity features are covered briefly in Chapter 12. See
  Chapter 5 for more details.

\item The minor differences include several match extensions, optional
  data typing on attributes, some new actions and functions, and other
  changes.
\end{itemize}

\subsection{Constructs Obsolete in RuleWorks}

None of the features described as obsolescent in the OPS5
Compatibility and Migration guide is implemented in
RuleWorks. Constructs that were obsolescent in OPS5 and are obsolete
in RuleWorks include:
\begin{itemize}
\item All of the \verb|OPS$...| API routines

\item CE variables

\item Value disjunctions or variables in the class position in a CE

\item \tt{LITERAL}, \tt{LITVAL}, and \tt{LITERALIZE}

\item \tt{SUBSTR} and \tt{VECTOR-ATTRIBUTE}

\item \tt{EXTERNAL} declaration

\item \tt{CALL} action and command

\item \tt{CBIND} action

\item \tt{HALT} action

\item \tt{COMPUTE} function

\item \tt{ACCEPT} function

\item \tt{ACCEPTLINE} function
\end{itemize}
      
\subsection{Features Added to RuleWorks}

The following differences between RuleWorks and OPS5 are side-effects
of the new modularity features:
\begin{itemize}
\item An \tt{ENTRY-BLOCK} (or \tt{DECLARATION-BLOCK}, or
  \tt{RULE-BLOCK}) construct and an \tt{END-BLOCK} construct are now
  required as wrappers around all other constructs.

\item There is a new compiler command with a completely new set of
  compiler qualifiers.

\item There is now no negated refraction.  The new refraction
  semantics are simply: ``Within an invocation of an entry-block, once
  an instantiation has fired, that specific instantiation will never
  fire again.''

\item \tt{MEA} is now the default conflict resolution strategy.

\item The conflict resolution strategy can only be set at compile time
  as an \tt{ENTRY-BLOCK} clause, so there is now no \tt{STRATEGY}
  command. An entry block checks when it starts up that each rule
  block it activates has the same strategy as it does.

\item There is no \tt{RUN} action, where in OPS5 it is allowed only in
  the \tt{STARTUP}. By default, an entry block always starts running.

\item The appearance of the command interpreter can now be controlled
  by the programmer using the Debug compiler command qualifier, the
  \tt{DEBUG} action, and the \verb|rul_debug| API routine. This is
  unlike OPS5 where source code changes had to be made to control the
  interpreter's appearance.

\item Because the same entry block can now be invoked multiple times
  recursively, when a rule instantiation is fired, it is not actually
  removed from the conflict set, even though it ceases to be visible
  there. If you are watching CS changes, be warned that you may see a
  fired instantiation leaving the conflict set long after it ceased to
  be visible via the CS command.

\item There is no \verb|^C| handler in the interpreter, as this could
  interfere with a user-supplied handler.

\item Direct instances of the class
  \verb|$ROOT| cannot be created in RuleWorks because of the data
  modularity.

\item The name \tt{MAIN} (or \verb,|main|,), when used as an entry
  block name without an \tt{ACCEPTS} clause or a \tt{RETURNS} clause
  for the \tt{ENTRY-BLOCK} creates a C compliant entry point named
  "main" by automatically adding a return type of long.

\item \tt{RETURN} is now a command as well as an action.

\item The extra-name argument to all the API routines that accept
  class names is now used. If supplied the extra-name should be the
  name of the block where the class was declared. This allows
  independent rule subprograms to each have a class defined that
  happen to have the same class name.

\item There is no \tt{RESTART} command.
\end{itemize}

\subsection{Minor Changes in RuleWorks}

This section covers differences between RuleWorks and OPS5 that were
not described as obsolescent in OPS5, and are not due to the new
RuleWorks modularity features.

\subsection{Match Extensions}

Several new matching features have been added, including:
\begin{itemize}
\item Variables and other value expressions are now evaluated within a
  value disjunctions, including variables bound in other condition
  elements.

\item Relational predicates (\verb|>|, \verb|<|, \verb|>=|, \verb|<=|,
  \verb|=|) can compare integers versus floats, and symbols versus
  symbols using localizable lexicographic ordering.

\item New similarity predicate, \verb|~=|, that for numbers compares
  integers and floats for equality within a 1\% margin, and for
  symbols uses the SOUNDEX algorithm to compare phonetic (in English)
  similarity.

\item New dissimilarity predicate, \verb|-~=|, that matches when the
  similarity test fails.

\item Comparisons of compound attributes and a \tt{COMPOUND} function
  are now supported.

\item The containment and non-containment predicates can now be
  applied to scalar attributes, as long as the argument to the
  predicate is a compound value (e.g.  \verb|^name [+]|). In addition,
  You can now use the containment and non-containment predicates in
  conjunction with a scalar predicate. This allows you to specify the
  test to be performed between a scalar value or attribute and each
  element of a compound attribute or value. Previously, the
  containment tests were always for identity.
\end{itemize}
      
\subsection{Optional Attribute Data Type Restrictions}

Optional attribute data types allow you to restrict the domain of an
attribute to a specific type of value. The set of supported domain
restrictions are: \tt{INTEGER}, \tt{SYMBOL}, \tt{FLOAT}, \tt{OPAQUE},
and \tt{INSTANCE-ID}. If the \tt{INSTANCE-ID} domain is specified,
then a specific class may optionally be identified by stating:
\tt{INSTANCE-ID OF}
 \verb|<class>|.

For example:
\begin{quote}
\begin{verbatim}
(object-class job-category
    ^category-name symbol
    ^max-salary float
    ^min-salary float
    ^employees compound instance-id of employee
    ^salaries compound float)
\end{verbatim}
\end{quote}
You are not required to supply attribute domain restrictions. If you
do, however, the compiler (and if necessary, the run-time system)
rigorously enforces those restrictions.

The \tt{PPCLASS} command shows any domain restrictions along with the
class's attributes.

\subsection{New RHS Actions}

The following RHS actions are new in RuleWorks
\begin{itemize}
\item The \tt{IF}\ldots\tt{THEN}\ldots\tt{ELSE}\ldots{} and
  \tt{WHILE}\ldots\tt{DO}\ldots{} actions provide conditional
  control over other RHS actions in rules or \tt{ON-} statements.
\item The \tt{REMOVE-EVERY} action provides a means to delete objects
  by class.
\end{itemize}
      
\subsection{New Functions}

The following functions are new in RuleWorks
\begin{itemize}
\item The \tt{EVERY} function (for RHS only) collects all instances of
  a class.  Accepts a class name, returns a compound of
  \tt{INSTANCE-ID}s.

\item The \tt{GET} function (for RHS only) can retrieve arbitrary
  attribute values from arbitrary instances. (Note: \tt{GET} can be
  used with unmatched objects. Unmatched objects are objects that were
  not in the firing instantiation. For example, a variable bound to
  the \verb|^foo| attribute, if it was an instance identifier, would
  be an unmatched object.)

\item The new \tt{GENINT} function returns a system-generated
  \tt{INTEGER}. The \tt{GENATOM} function changed to take an optional
  argument that specifies a prefix for the generated symbol. The
  default prefix is still \verb|G:|.

\item The \tt{IS-OPEN} function checks whether a file stream is
  open. It accepts a \it{file-id}; returns \tt{OUT} or \tt{IN} if open, \tt{NIL}
  if not.

\item The \tt{MAX} and \tt{MIN} functions find the largest or smallest
  value among their arguments.

\item The \tt{SUBCOMPOUND} function (and the new \tt{SUBSYMBOL}) can
  be used on the LHS.

\item The \tt{SUBSYMBOL} function extracts a segment of a symbol.

\item The \tt{SYMBOL-LENGTH} function counts the characters in a
  symbol. Accepts a symbolic value, returns an integer.
\end{itemize}
    
\subsection{Miscellaneous Changes}

\begin{itemize}
\item As a result of the introduction of the new semantics for the
  relational predicates, the predicate, \verb|=| (now called the equality
  predicate), is no longer equivalent to the implied identity
  predicate, and it cannot bind an unbound variable.

\item The implied predicate is now identity, not equality. The
  identity predicate can bind a variable.

\item Definition of equality for symbols has changed: it is no longer
  case-sensitive.

\item Containment predicate has changed: you can now test whether a
  scalar value is contained by a compound, as well as whether a
  compound value contains a scalar.

\item Variables inside disjunctions of values are evaluated in
  RuleWorks.

\item First argument to \tt{COPY}, \tt{MAKE}, \tt{MODIFY} and
  \tt{SPECIALIZE} actions is no longer required to be a variable bound
  to a \verb|^$ID| attribute on the LHS. Can be bound to a different
  attribute or the result of a \tt{GET} function, but must evaluate to
  an \tt{INSTANCE-ID} of an object class that is visible to the active
  entry block.

\item External routine names are not allowed to be the same as actions
  or construct types.

\item External routines can accept a dynamically-size array.

\item Rules or classes can not have names that are predicate operators
  or non-symbolic values.

\item The name of the \tt{WATCH} command has been changed to
  \tt{TRACE}.

\item The output from \tt{PPWM}, \tt{WM}, and \tt{TRACE} \tt{WM} now shows the
  readforms of all attribute values, instead of the printforms. The
  output from \tt{WRITE} still shows the printforms.

  The syntax for many of the debugging commands has changed. Using the
  old syntax now generates a syntax error but also usage and example
  messages.

\item The behavior of the \tt{EXIT} command has changed since
  OPS5. Previously, if the OPS5 routine was not your main program,
  \tt{EXIT} would return control to the calling program. In RuleWorks,
  \tt{EXIT} always returns to the operating system.

\item Unmatched objects can now be acted upon by the \tt{REMOVE},
  \tt{COPY}, \tt{MODIFY}, and \tt{SPECIALIZE} actions. The compiler
  gives an informational message (and generates extra run-time
  checking code) whenever it is unable to verify that the instance
  argument passed to one of these actions is indeed a valid instance
  identifier of an existing object.

\item Two new special atoms; one for each of the new types: the null
  Instance-id, \verb|#0|, and the null Opaque value, \verb|%x0|
  (equivalent to C's \verb|NULL|).
\end{itemize}
      
%%% Local Variables:
%%% mode: latex
%%% TeX-master: "rwug"
%%% End:

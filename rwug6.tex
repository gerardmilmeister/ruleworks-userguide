\chapter{Using RuleWorks with Other Languages}
\label{c:otherlang}

Programming tasks such as computing mathematical expressions,
manipulating strings, and editing large quantities of data are often
easier and more efficient to develop in languages other than
RuleWorks. If you are developing a RuleWorks program that needs to
perform these types of tasks, you should consider using external
routines. An external routine is a function or subroutine written in a
language other than RuleWorks.  External routines include prewritten
routines such as system services and run-time library (RTL) routines.

RuleWorks allows external routines that accept arguments and return
results according to the hardware platform's calling standard. This
means that you can call external routines that are not written
specifically for RuleWorks. Example~\ref{e:6-1} shows a simple
RuleWorks program calling the C RTL routine, \verb|sqrt|.

\begin{example}[h]
\begin{quote}
\begin{verbatim}
(declaration-block decls)

(object-class start)

(end-block decls)

(entry-block main
    (uses decls)

(on-entry
    (make start)))

(external-routine sqrt
    (accepts double-float)
    (returns double-float))

(rule square-root
    (start)
  -->
    (write |Enter a number> |)
    (bind <input> (accept-atom))
    (write |The square root is| (sqrt <input>)))

(end-block main)
\end{verbatim}
\end{quote}
\caption{Calling an External Routine from RuleWorks}
\label{e:6-1}
\end{example}

RuleWorks also allows entry blocks to be called from other
languages. Entry blocks can accept arguments and return a value. Entry
blocks can call other entry blocks, but the callers must declare the
other entry blocks with \tt{EXTERNAL-ROUTINE} declarations. Entry
blocks can even call themselves recursively.

\section{Calling External Routines from RuleWorks}

You must declare external routines before you call them in your
RuleWorks program.  External routines are scoped to the block in which
they are declared.

The easiest way to ensure that the correct declarations have been made
is to place the \tt{EXTERNAL-ROUTINE} declarations in a separate
declaration block and have each module acquire the compiled
declarations via a \tt{USES} clause. See Chapter~\ref{c:program} for
details on data partitioning and declaration sharing.

The complete syntax for the \tt{EXTERNAL-ROUTINE} declaration is shown
in Figure~\ref{f:6-1}:

\begin{figure}[h]
\begin{quote}
\tt{(EXTERNAL-ROUTINE} \it{routine-name}\par
\qquad[\tt{(ALIAS-FOR} actual-routine-name\tt{)}]\par
\qquad\qquad[\verb|<|\it{formal-parameter-name}\verb|>| [\it{size}]]\par
\qquad[\tt{(ACCEPTS} \{\it{external-type}\} \ldots\verb|)|]\par
\qquad\qquad[\it{passing-mechanism}]\par
\qquad\qquad[\verb|<|\it{formal-parameter-name}\verb|>| [\it{size}]]\par
\qquad[\tt{(ACCEPTS} \{\it{external-type}\}\tt{)}]\}\par
\qquad\qquad[\it{passing-mechanism}]\par
\end{quote}
\caption{External Routine Declaration Syntax}
\label{f:6-1}
\end{figure}

The routine-name is required; the \tt{ALIAS-FOR}, \tt{ACCEPTS} and
\tt{RETURNS} clauses are optional.  The syntax for calling an external
routine is shown below:

\begin{quote}
\verb|(|\it{routine-name} [\{\it{value-expression}\} \ldots]\verb|)|
\end{quote}

\tt{EXTERNAL-ROUTINE} declarations may appear inside any type of block, but
they must appear inside of some kind of block, and they must appear
before they are used. Duplicate declarations of the same external
routine will generate a warning and be ignored so long as the
declarations are identical.

External routines that return a value can be used on the RHS (as shown
in Example~\ref{e:6-1}) or on the LHS. External routines that do not return a
value must be used on the RHS as if they were RuleWorks actions. You
can also call an external routine that does return a value as if it
were an RHS action, but in this case RuleWorks ignores the return
value.

\subsubsection{Writing Portable Code}

The \tt{ALIAS-FOR} clause allows you to declare that the routine name
used inside RuleWorks is not the actual name that will be linked. This
is useful for mapping a case-sensitive external routine name onto a
case-insensitive RuleWorks symbol.

The actual function name must be quoted to preserve case. For example:

\begin{quote}
\begin{verbatim}
(external-routine xt_parent ; routine name used inside TINpan
    (alias-for |XtParent|) ; actual function name
    (accepts pointer)
    (returns pointer))
\end{verbatim}
\end{quote}

You can use at most one \tt{ALIAS-FOR} clause in an \tt{EXTERNAL-ROUTINE}
declaration. You must provide exactly one function name in an
\tt{ALIAS-FOR} clause.

\subsubsection{Passing Parameters}

Use the \tt{ACCEPTS} clause to declare one or more parameters for an
external routine. The \verb|<|\it{formal-parameter-name}\verb|>| is
optional; use it to help make your code more self-documenting. The
external-type-name is required. The passing-mechanism for each
argument is also optional; the default mechanism for each external
type is shown in Table~\ref{t:6-1}. Arrays of each type are also
allowed, passed \tt{BY REFERENCE} only.

\begin{table}[h]
  \begin{tabularx}{\columnwidth}{lXXX}
    \toprule
    & \multicolumn{3}{c}{Argument Passing Mechanisms} \\
    \midrule
    External Data Type & \tt{BY VALUE} & \tt{BY REFERENCE READ-ONLY} & \tt{BY REFERENCE READ-WRITE} \\ 
    \tt{BYTE} & Default & * & * \\
    \tt{SHORT} & Default & * & * \\
    \tt{LONG} & Default & * & * \\
    \tt{UNSIGNED-BYTE} & Default & * & * \\
    \tt{UNSIGNED-SHORT} & Default & * & * \\
    \tt{UNSIGNED-LONG} & Default & * & * \\
    \tt{SINGLE-FLOAT} (1) & Default  &      *       &      * \\
    \tt{DOUBLE-FLOAT} (2) & Default  &      *       &      * \\
    \tt{ASCIZ} & N/S      &   Default    &      * \\
    \tt{ASCID} & N/S      &   Default    &      * \\
    \tt{POINTER} (3) & Default  &      *       &      * \\
    \tt{ATOM} (4) & Default  &      *       &      * \\
    \bottomrule
  \end{tabularx}
  \begin{quote}
    (1) On VMS systems, \tt{SINGLE-FLOAT} refers to
    \verb|F_float_data|

    (2) On VAX VMS systems, \tt{DOUBLE-FLOAT} refers to
    \verb|D_float data|; on OpenVMS for Alpha AXP systems,
    \tt{DOUBLE-FLOAT} refers to \verb|G_float| data.

    (3) Opaque virtual address

    (4) Opaque atom

    * Supported, but not the default

    N/S Not supported
  \end{quote}
  \caption{External Data Types and Argument-Passing Mechanisms}
  \label{t:6-1}
\end{table}

Numeric and pointer external types default to zeros. \tt{ASCIZ} and
\tt{ASCID} externals default to the zero-length string. The \tt{ATOM}
external defaults to \tt{NIL}.

RuleWorks's compound values correspond to an array of atoms in
external routines. You declare a parameter as an array by putting
brackets (\verb|[]|) between the formal parameter name and the external
data type. If you do not declare the size of the array by putting an
integer between the brackets, then the array received by the external
routine has as many elements as the compound value had when it was
passed out, and the external routine cannot change the size of the
array. An array returned by an external routine must be the declared
size.

Since the size of the array is not automatically passed, applications
using empty brackets must define a convention such as a specific value
to signal the end of the array, or pass the length as a separate
parameter. Example~\ref{e:6-2} shows a C program that passes a
compound value.

\begin{example}[!h]
\renewcommand{\baselinestretch}{0.9}
\begin{quote}
\begin{verbatim}
#include <stdio.h>
#include <rul_rtl.h>

/* Function: concat_compound
 *
 * Accepts: The length of the array passed as the second argument
 *   An array of asciz strings
 * Function: Constructs a string containing the values in a compound.
 * Example:
 *   Given arguments: 3 and ("A","B","C")
 *   it returns the string "A-B-C"
 * Side Effect:
 *   Prints out all the input strings in the second argument,
 *   assuming that they all fit into one symbol.
 * Returns
 *   The string formed by concatenating all the elements of the
 *   compound, assuming that they will all fit into a symbol.
 *   If they do not fit into a symbol, the result is truncated
 *   at the maximum symbol size.
 */
char *concat_compound (long num_elements, char *az_array[])
{
    int  i, index, len;
    static char result[RUL_C_MAX_SYMBOL_SIZE+1];
    index = 0;
    printf ("\n Elements found in compound:");
    for ( i=0; i<num_elements; i++) {
        len = strlen (az_array[i]);
        if (len > RUL_C_MAX_SYMBOL_SIZE - index)
            len = RUL_C_MAX_SYMBOL_SIZE - index;
        strncpy (&result[index], az_array[i], len);
        index = index + len;
        if (index >= RUL_C_MAX_SYMBOL_SIZE) {
            /* not enough space in a symbol for all the compound values */
            result[RUL_C_MAX_SYMBOL_SIZE] = '\0';
            return (&result);
        }
        if ((i + 1) < num_elements) {
            /* insert a dash between compound values */
            result[index] = '-';
            index = index + 1;
        }
        printf ("\n %s", az_array[i]);
    }
    if (index < RUL_C_MAX_SYMBOL_SIZE+1) {
        result[index] = '\0';
    }
    return (&result);
}
\end{verbatim}
\end{quote}
\caption{Passing a Compound Value: C Function}
\label{e:6-2}
\end{example}

Example~\ref{e:6-3} shows the RuleWorks program that calls the C function in
Example~\ref{e:6-2}.

\begin{example}[h]
\begin{quote}
\begin{verbatim}
(entry-block main

(object-class classname ^comp-attr compound ^length ^atom)

(external-routine concat_compound
    (accepts <comp_len> long by value
             <az_array> [] asciz by reference read-only)
    (returns <ret_asciz> asciz))

(external-routine strlen
    (accepts <az_string> asciz)
    (returns <length> long))

(on-entry
    (make classname ^comp-attr (compound a b c)))

(rule call-C-function
    (classname ^$id <obj> ^atom <sym> ^length <> (strlen <sym>)
         ^comp-attr <comp>)
  -->
    (bind <result> (concat_compound (length <comp>) <comp>))
    (write (crlf) | | <comp> |==>| <result>)
    (remove <obj>))

(end-block main)
\end{verbatim}
\end{quote}
\caption{Passing a Compound Value: RuleWorks Program}
\label{e:6-3}
\end{example}

The dialog in the following example illustrates compiling, linking,
and running Example~\ref{e:6-2} and Example~\ref{e:6-3} on a VMS
system.

\begin{quote}
\begin{verbatim}
$rulework concat
...
$cc concat_comp
$link concat,concat_comp,rul$library:rul_rtl/lib
$run concat
Elements found in compound:
     A
     B
     C
     A B C ==> A-B-C
\end{verbatim}
\end{quote}

\subsubsection{Passing Non-Atomic RuleWorks Objects}

In RuleWorks, there are several kinds of objects with no corresponding
entity in external routines. These objects cannot be passed directly,
but they can be passed indirectly. They are listed below with the
indirect mechanism by which they can be passed.

\begin{table}[h]
  \begin{tabularx}{\columnwidth}{lX}
    \toprule
    Object & Indirect Passing Mechanism \\
    \midrule
    WMO & By object identifier, using the external type
          \tt{POINTER} or \tt{ATOM} \\
    Compound value & By converting it into an array \\
    \bottomrule
  \end{tabularx}
  \caption{Passing Non-Atomic Objects}
  \label{t:6-2}
\end{table}

\subsubsection{External Data Types}

Values in RuleWorks are converted into external data types whenever
they are passed to any external routine. These conversions are done
automatically based on the \tt{EXTERNAL-ROUTINE} declarations. The
type specifications are checked at compile time for constants and at
run time for expressions. Table~\ref{t:6-3} shows which RuleWorks
types can be passed for each external type.

\begin{table}[h]
  \begin{tabularx}{\columnwidth}{l|lllll}
    \toprule
    & \multicolumn{5}{c}{RuleWorks Type} \\
    \midrule
    External Type   & \tt{INTEGER}    & \tt{FLOAT} & \tt{SYMBOL} & \tt{INSTANCE-ID} & \tt{OPAQUE} \\
    \midrule
    \tt{BYTE}            & Natural    & Implicit   & Error      & Error       & Error   \\
    \tt{SHORT}           & Natural    & Implicit   & Error      & Error       & Error   \\
    \tt{LONG}            & Equivalent & Implicit   & Error      & Error       & Error   \\
    \tt{UNSIGNED-BYTE}   & Natural    & Implicit   & Error      & Error       & Error   \\
    \tt{UNSIGNED-SHORT}  & Natural    & Implicit   & Error      & Error       & Error   \\
    \tt{UNSIGNED-LONG}   & Natural    & Implicit   & Error      & Error       & Error   \\
    \tt{SINGLE-FLOAT} (1) & Implicit   & Natural    & Error      & Error       & Error   \\
    \tt{DOUBLE-FLOAT} (2) & Implicit   & Equivalent & Error      & Error       & Error   \\
    \tt{ASCID}           & Implicit*  & Implicit*  & Equivalent & Implicit*   & Error   \\
    \tt{ASCIZ}           & Implicit*  & Implicit*  & Equivalent & Implicit*   & Error   \\
    \tt{POINTER}         & Error      & Error      & Error      & Error       & Natural \\
    \tt{ATOM} & \multicolumn{5}{c}{No conversion required for \tt{ATOM}s} \\
    \bottomrule
  \end{tabularx}
  \begin{quote}
    (1) On VMS systems, the external type \tt{SINGLE-FLOAT} refers to
    \verb|F_float| data

    (2) On VAX VMS systems, the external type \tt{DOUBLE-FLOAT} refers
    to \verb|D_float data|; on OpenVMS for Alpha AXP systems, to
    \verb|G_float|.

    * \tt{ASCID} and \tt{ASCIZ} values are coerced outbound only. All
    other entries in this table apply both to calling out from and
    calling in to RuleWorks
  \end{quote}
  \caption{Type Conversions of External Routine Parameters}
  \label{t:6-3}
\end{table}

The ``natural'' type conversion referred to in Table~\ref{t:6-3} is
the one used when external data is returned to RuleWorks in a
\tt{READ-WRITE} parameter. The ``equivalent'' RuleWorks type/external
type pairs have no loss of precision when passed either way as a
\tt{READ-WRITE} parameter. The ``implicit'' type conversions are
handled automatically by RuleWorks according to the declarations. For
example, you can pass an \tt{INTEGER} atom as a \tt{SINGLE-FLOAT},
\tt{DOUBLE-FLOAT}, \tt{ASCIZ}, or \tt{ASCID} parameter without using
the \tt{FLOAT} or \tt{SYMBOL} conversion functions. However, passing a
\tt{SYMBOL} or \tt{INSTANCE-ID} as any numeric external type causes an
error.

``Outbound only'' means that when an \tt{INTEGER}, \tt{FLOAT}, or
\tt{INSTANCE-ID} atom is passed from RuleWorks to an external routine
that expects a string, the value is coerced.  However, if that string
is returned (as a \tt{READ-WRITE} parameter), it becomes an atom of type
\tt{SYMBOL}.

A compound value can be passed only as an array, and only a compound
can be passed as an array. Each atom within the compound must be
compatible (according to Table~\ref{t:6-3}) with the external type.

If the external type is not large enough to represent the value being
passed out, a warning is signaled. For example, 300 should not be
passed out to an external routine that expects a byte.

Example~\ref{e:6-5} and Example~\ref{e:6-6} show how to return a
compound value, and how to use a \tt{READ-WRITE} parameter.

\begin{example}{!h}
\begin{quote}
\begin{verbatim}
#include <stdio.h>
#include <rul_rtl.h>

/* Example function explode
 *
 * Accepts:
 * An asciz string
 * Function:
 * Turns a symbol into an array of characters.
 * Example:
 * Given the argument "HELLO"
 * Writes 5 into the second argument and
 * Returns "H", "E", "L", "L", "O", "", "", ...
 * Side Effect:
 * Modifies the write-only argument, num_returned.
 * Returns
 * An array (rul_c_max_symbol_size in length)
 * of very short ASCIZ strings
 * (at most one character each).
 */

char **explode (char *in_string, long *num_returned)
{
    /* The actual string space */
    static char short_strings[RUL_C_MAX_SYMBOL_SIZE][2] ;
    /* The array of string pointers to be returned */
    static char *exploded[RUL_C_MAX_SYMBOL_SIZE] ;
    static long called_before = FALSE ;
    long i ;
    if (! called_before) {
        /*
        ** On the first invocation of this function, set up
        ** the array of string pointers.
        */
        for (i=0; i<RUL_C_MAX_SYMBOL_SIZE; i++) {
            short_strings[i][1] = '\0' ;
            exploded[i] = &(short_strings[i][0]) ;
        }
        called_before = TRUE ;
    }

    /*
    ** For each character in the input string, create an
    ** entry in the array of strings to be returned.
    */
    *num_returned = strlen(in_string) ;
    for (i=0; i<RUL_C_MAX_SYMBOL_SIZE; i++) {
        if (i < *num_returned) {
        short_strings[i][0] = in_string[i] ;
        } else {
        short_strings[i][0] = '\0' ;
        }
    }
    return (exploded) ;
}
\end{verbatim}
\end{quote}
\caption{External Function That Returns an Array}
\label{e:6-5}
\end{example}


\begin{example}{!h}
\begin{quote}
\begin{verbatim}
(entry-block main)

(external-routine explode
    (alias-for |explode|)
    (accepts <string> asciz
             <count> long by reference read-write)
    (returns <strings>[256] asciz))
 
(object-class name ^as-word ^as-compound compound)

(on-entry
    (make name ^as-word abc)
    (make name ^as-word |hello|)))

(rule explode-it
    (name ^$id <n-id> ^as-word { <> NIL <word> } ^as-compound [=] 0)
  -->
    (bind <ret-len> 0)
    (bind <ret-list> (explode <word> <ret-len>))
    (modify <n-id> ^as-compound (subcompound <ret-list> 1 <ret-len>)))

(end-block main)
\end{verbatim}
\end{quote}
\caption{RuleWorks Program That Passes a READ-WRITE Parameter}
\label{e:6-6}
\end{example}

The dialog in Example~\ref{e:6-7} shows what happens when you run the
RuleWorks program in Example~\ref{e:6-6}.

\begin{example}{!h}
\begin{quote}
\begin{verbatim}
RuleWorks> trace on wm
RuleWorks> run 2
<=WM: #2 2 [NIL] (NAME ^AS-WORD hello)
=>WM: #2 3 [EXPLODE-IT] (NAME ^AS-WORD hello ^AS-COMPOUND (COMPOUND h e l l o))
<=WM: #1 1 [NIL] (NAME ^AS-WORD ABC)
=>WM: #1 4 [EXPLODE-IT] (NAME ^AS-WORD ABC ^AS-COMPOUND (COMPOUND A B C))
%RUL-I-PAUSE, Pausing after running requested number of rules
RuleWorks>
\end{verbatim}
\end{quote}
\caption{Passing a \tt{READ-WRITE} Parameter}
\label{e:6-7}
\end{example}

\begin{note}
  Whenever a \tt{SYMBOL} is converted to a string to be passed to an
  external routine, the string passed contains the print form of the
  symbol. Thus, in Example~\ref{e:6-6}, the quoted symbol, \verb,|hello|,, is
  returned as lowercase letters but the unquoted symbol, \tt{abc}, is
  returned as uppercase letters.
\end{note}

\subsection{Passing Mechanisms}

Declaring the passing mechanism for each parameter is optional. If you
do not declare a passing mechanism, RuleWorks uses the default
appropriate to the external data type of the parameter: \tt{BY VALUE}
for all types exceptstrings, whose default is \tt{BY REFERENCE
  READ-ONLY}. The last three columns in Table~\ref{t:6-1} shows the
default, supported, and unsupported passing mechanisms for each
external data type.

For \tt{ACCEPTS} arguments, RuleWorks provides three passing
mechanisms:

\begin{itemize}
\item \tt{BY VALUE}

  Passes a copy of the value of the argument. Any changes to the
  argument by the external routine are ignored when the external
  routine returns.

\item \tt{BY REFERENCE READ-ONLY}

  Passes a pointer to a copy of the argument. Any changes to the
  argument by the external routine are ignored when the external
  routine returns.

\item \tt{BY REFERENCE READ-WRITE}

  Passes a pointer to a copy of the argument. In the typical case,
  where the argument passed BY \tt{REFERENCE READ-WRITE} is a bound
  variable, then after the external routine returns the variable is
  bound to the value written by the external routine.

  Normally, external routines with \tt{READ-WRITE} passing mechanisms
  are used on the RHS of rules, and each \tt{READ-WRITE} argument
  passed is a bound variable.  Passing an unbound variable on the RHS,
  or passing any variable on the LHS, or passing the result of an
  expression, causes the value set by the external routine to be
  ignored and generates a compiler warning.

  A symbol that you pass out \tt{BY REFERENCE READ-WRITE} as external
  type \tt{ASCIZ} is passed in a buffer of
  \verb|RUL_C_MAX_SYMBOL_SIZE| characters. The external routine is
  free to fill the buffer with a string of up to
  \verb|RUL_C_MAX_SYMBOL_SIZE| characters. (Symbols that you pass as
  \tt{ASCID} or \tt{ASCIZ BY REFERENCE READ-ONLY} are only as long as
  needed to hold the value being passed.)

  If the actual parameter is a compound bound to a variable passed
  \tt{BY REFERENCE READ-WRITE}, changes made to elements of the array
  by the external routine will be reflected in the values of the
  compound variable. The number of elements in the compound value
  cannot be changed by the external routine.

  If the number of atoms in the actual compound value being passed is
  greater than the number of elements in the array, the excess atoms
  are not passed and a warning is given. If there are fewer atoms than
  array elements, the array is padded with a default value for that
  external type. Numeric external types default to zeros. \tt{ASCIZ}
  and \tt{ASCID} externals default to the zero-length string. The
  \tt{POINTER} external type defaults to \tt{NULL}. The \tt{ATOM}
  external type defaults to \tt{NIL}.

  For example, the RuleWorks program in Example~\ref{e:6-6} calls an external
  function to provide a value for a \tt{BIND} action. The external
  function, shown in Example 6-5, takes a symbol and returns an array
  that contains the characters in that symbol. The external function
  also returns the number of characters in the array in the
  \tt{READ-WRITE} argument \verb|<COUNT>|.

  Some system services require that you omit some \tt{BY REFERENCE}
  parameters. An empty pair of parentheses \verb|()| at the calling
  site causes 0 to be passed by value, for \tt{BY REFERENCE}
  parameters.
\end{itemize}

\subsubsection{Order of Argument Evaluation}

You should not depend on the order of evaluation or of side effects
that result from evaluation of functions in calls to external
routines, or anywhere else. For instance, if Example~\ref{e:6-6} used
the following rule, the program would not work:

\begin{quote}
\begin{verbatim}
(rule does-not-explode-it
    (name ^$ID <n-id> ^as-word { <> NIL <word> } ^as-compound [=] 0)
  -->
    (bind <ret-len> 0)
        (modify <n-id>
        ^as-compound
            (subcompound (explode <word> <ret-len>) 1 <ret-len>)))
\end{verbatim}
\end{quote}

To avoid dependence on the order of evaluation of arguments, bind all
the argument expressions that include function calls, except the last,
before you call the external routine.

\subsubsection{Type Changes to Arguments}

When variables are passed \tt{BY REFERENCE READ-WRITE}, two data type
conversions are performed. The first conversion is from the RuleWorks
atom to the specified external type. The second is from the specified
external type to the natural RuleWorks type (see Table~\ref{t:6-3}),
which may or may not be the original type.

For example, if the variable being passed \tt{BY REFERENCE READ-WRITE}
is bound to an \tt{INTEGER} atom, and the external type is
\tt{DOUBLE-FLOAT}, the variable is rebound to a \tt{FLOAT} atom after
the routine returns. If the variable being passed \tt{BY REFERENCE
  READ-WRITE} is bound to an \tt{INSTANCE-ID} atom, and the external
type is \tt{ASCIZ}, the variable is rebound to a \tt{SYMBOL} atom
after the routine returns.

If the variable is bound to a compound containing some float atoms and
some integer atoms, and the external type is \tt{DOUBLE-FLOAT}, the
variable is bound to a compound containing only float atoms after the
routine returns. If the variable is bound to a compound containing
some float atoms, some symbol atoms, and some integer atoms, and the
external type is \tt{ASCID}, the variable is bound to a compound containing
only symbol atoms after the routine returns.

\subsubsection{Visibility of Changes to Arguments}

When variables bound on the LHS are passed out \tt{BY REFERENCE
  READ-WRITE} on the RHS, the called routine can modify them. However,
when the called routine returns, the changes are reflected into the
bound variables, but not in the WMO attributes from which those
variables were originally set. If the new values need to be reflected
back into changes in the object, you must explicitly modify the
object.

\subsubsection{Returning a Value to RuleWorks}

Use the \tt{RETURNS} clause to declare an external routine as a
function that returns a value. The
\verb|<|\it{formal-parameter-name}\verb|>| and the passing-mechanism
for the return value are optional; the \it{external-type-name} is required.

For the \tt{RETURNS} clause, only two passing-mechanisms are defined:
\tt{BY VALUE} and \tt{BY REFERENCE}. No access mechanisms are defined
for return values.

RuleWorks automatically converts return values from their external
data types to the appropriate RuleWorks data types. The following
example shows an \tt{EXTERNAL-ROUTINE} declaration for the C language
library function that finds the length of a string:

\begin{quote}
\begin{verbatim}
(external-routine strlen
    (accepts <string> asciz)
    (returns <length> long))
\end{verbatim}
\end{quote}

Given this declaration, the following RHS action first converts the
symbol \tt{CHARLIE} into an \tt{ASCIZ} string by using the symbol's
print form, calls the ``\verb|strlen|'' function passing the new \tt{ASCIZ}
string, and then converts the \tt{LONG} return value into a RuleWorks
\tt{INTEGER}:

\begin{quote}
\begin{verbatim}
(modify <the-wmo> ^length (strlen charlie))
\end{verbatim}
\end{quote}

Empty brackets are not valid in the \tt{RETURNS} clause. You must
explicitly declare the size of an array that is returned to RuleWorks,
as shown in Example~\ref{e:6-6}.

When returning a value for which memory must be allocated to store the
actual value(s) (that is, a string or an array), the external routine
is responsible for both the allocation and deallocation of that
memory. In Example~\ref{e:6-5} the external routine declares the memory for
the return values as \tt{STATIC}, thereby allocating once and reusing
the same memory each time the routine is invoked.

\section{Using RuleWorks Run-Time Library Routines}

RuleWorks provides a library of callable run-time routines that allow
you to access working memory from your external
routines. Table~\ref{t:6-5} through Table~\ref{e:6-10} list all the
RTL routines; detailed descriptions of each routine are provided in
Chapter~\ref{c:api}.

The RTL actually includes multiple implementations of each routine,
one for each NAS binding.

\subsection{Choosing Bindings}

The RuleWorks RTL follows the Network Application Support (NAS)
guidelines for a portable programming interface: RuleWorks supplies
the VMS calling standard bindings for VMS systems, the C bindings for
all platforms, and the f77 bindings for UNIX systems. The following
three sections summarize the different bindings.

\begin{note}
  All the name changes are done automatically by the compilers
  involved.
\end{note}

\subsubsection{Using the VAX Bindings}

The VMS calling standard bindings are supplied on VMS systems to
support the VMS high-level languages (such as VAX FORTRAN and VAX
Pascal). These bindings use the VMS Procedure Calling Standard, as
follows:

\begin{itemize}
\item Actual routine names are as shown in this guide, except that
  they are all uppercase letters.
\item String arguments are received by VMS descriptor.
\item All other scalar data types are received by reference.
\item Arrays of any type are received by reference.
\end{itemize}

\subsubsection{Using the C Bindings}

The C bindings are supplied on all platforms to support languages such
as C and C++. The C bindings on UNIX systems also support Pascal. The
routines in this binding follow C conventions, as listed below:

\begin{itemize}
\item Actual routine names are exactly as shown in this guide,
  including case.
\item String arguments are received as pointers to null-terminated
  strings (\tt{ASCIZ}).
\item All other scalar data types are received by value.
\item Arrays of any type are received by reference.
\end{itemize}

\subsubsection{Using the f77 Bindings}

The f77 bindings are supplied on UNIX systems to support the current
f77 calling conventions, as follows:

\begin{itemize}
\item Actual routine names are entirely in lowercase letters and as
  shown in this guide. (The f77 compiler automatically adds a trailing
  underscore.)
\item String arguments are received by reference and additional hidden
  arguments, specifying the length of each string (received by value),
  are automatically appended to the argument list.
\item All other scalar data types are received by reference.
\item Arrays of any type are received by reference.
\end{itemize}

\subsection{Declaring RTL Routines}

You must declare the RuleWorks RTL routines in the language you are
using. On VMS and UNIX platforms, RuleWorks provides a number of
include files that contain the necessary declarations (see
Table~\ref{t:6-4}). On other platforms, the only include file is
\verb|rul_rtl.h| for use with C and C++.

\begin{table}[h]
  \begin{tabularx}{\columnwidth}{lll}
    \toprule
    Language & VMS File & UNIX File \\
    \midrule
    Ada      & \verb|RUL$LIBRARY:RUL_RTL.ADA| & \verb|/usr/lib/cmplrs/rulework/rul_rtl.ada| \\
    BASIC    & \verb|RUL$LIBRARY:RUL_RTL.BAS| & \verb|/usr/lib/cmplrs/rulework/rul_rtl.bas| \\
    C        & \verb|RUL$LIBRARY:RUL_RTL.H|   & \verb|/usr/include/rul_rtl.h|               \\   
    FORTRAN  & \verb|RUL$LIBRARY:RUL_RTL|     & \verb|/usr/lib/cmplrs/rulework/rul_rtl.for| \\
    Pascal   & \verb|RUL$LIBRARY:RUL_RTL.PAS| & \verb|/usr/include/pascal/rul_rtl.h|        \\
    PL/I     & \verb|RUL$LIBRARY:RUL_RTL.PLI| & \verb|/usr/lib/cmplrs/rulework/rul_rtl.pli| \\
    BLISS*32 & \verb|RUL$LIBRARY:RUL_RTL.R32| & \verb|/usr/lib/cmplrs/rulework/rul_rtl.r32| \\
  \end{tabularx}
  \caption{Include Files Provided by RuleWorks}
  \label{t:6-4}
\end{table}

For example, if you are interfacing a VAX Pascal program to RuleWorks,
you use a \verb|%INCLUDE| directive to place the \verb|RUL_RTL.PAS|
file in your program as follows:

\begin{quote}
\begin{verbatim}
{ Include RuleWorks routine declarations }
%INCLUDE 'RUL$LIBRARY:RUL_RTL.PAS'
\end{verbatim}
\end{quote}

For a VAX BASIC external routine to access \verb|the RUL_RTL.BAS|
declarations, place the following \verb|%INCLUDE| statement in the
routine:

\begin{quote}
\begin{verbatim}
%INCLUDE "RUL_RTL.BAS"
\end{verbatim}
\end{quote}

\begin{table}[h]
  \begin{tabularx}{\columnwidth}{lX}
    \toprule
    RTL Routine & Description \\
    \midrule
    \verb|rul_get_attr_atom| & Returns the value of a scalar attribute. \\
    \verb|rul_get_class_string| & Returns the class name of an object. \\
    \verb|rul_get_class_string_length|  & Returns the number of characters 
                                          in a class name. \\
    \verb|rul_get_comp_attr_length| & Returns the number of elements in a compound
                                      attribute value. \\
    \verb|rul_get_comp_attr_string| & Returns the read forms of all the values in a
                                      compound attribute. \\
    \verb|rul_get_comp_attr_string_len| & Returns the number of characters in the
                                          read form of a compound attribute value. \\
    \verb|rul_get_comp_elem_atom| & Returns the value of a single element of a
                                    compound attribute. \\
    \verb|rul_get_instance| & Returns the read form of an object. \\
    \verb|rul_get_instance_length| & Returns the number of characters in the
                                     read form of an object. \\
    \verb|rul_get_next_instance| & Allows iteration over working memory. \\
    \bottomrule
  \end{tabularx}
  \caption{RTL Routines for Accessing Working Memory}
  \label{t:6-5}
\end{table}

Example~\ref{e:6-8} and Example~\ref{e:6-9} accept an \tt{INSTANCE-ID}
and prints that object; makes a new object with
\verb|rul_make_instance|, modifies an attribute with
\verb|rul_set_attr_string|, and prints the result; another new object
with \verb|rul_copy_instance|, modifies it, and prints it; and finally
removes the object created with \verb|rul_make_instance|.

\begin{example}[!h]
\begin{quote}
\begin{verbatim}
(entry-block main)

(external-routine mess_with (accepts atom))

(object-class person ^name ^called)

(on-entry
    (bind <obj> (make person ^name |George| ^called friend))
    (mess_with <obj>))

(end-block)
\end{verbatim}
\end{quote}
\caption{Changing Working Memory: RuleWorks Program}
\label{e:6-8}
\end{example}

Running Example~\ref{e:6-8} produces the following output:

\begin{quote}
\begin{verbatim}
Step 1: (PERSON ^$ID #1 ^NAME |George| ^CALLED FRIEND)
Step 2: (PERSON ^$ID #2 ^NAME |George| ^CALLED |Neighbor|)
Step 3: (PERSON ^$ID #3 ^NAME |George| ^CALLED TROUBLE)
Removed Instance: #2
\end{verbatim}
\end{quote}

\begin{example}[h]
\renewcommand{\baselinestretch}{0.9}
\begin{quote}
\begin{verbatim}
#include <stdio.h>
#include <rul_rtl.h>
/* Set BUFF_SIZE big enough to store the printform
   of any of our working-memory objects */
#define BUFF_SIZE 1000
/* Set ID_BUFF_SIZE big enough for any symbol's printform */
#define ID_BUFF_SIZE RUL_C_MAX_SYMBOL_SIZE*2
void MESS_WITH (rul_atom wme_id)
{
    char obj_string_buffer[BUFF_SIZE];
    char id_string_buffer[ID_BUFF_SIZE];
    long b_len;
    rul_atom new_wme_id, a_wme_id;
    /* Verify that the argument is an instance id */
    if (rul_atom_is_instance_id (wme_id)) {
        /* Verify that there exists a working memory element
           with the given instance id */
        if (rul_is_instance (wme_id)) {
            /* Print the read form of the working memory object */
            b_len = rul_get_instance (obj_string_buffer, BUFF_SIZE, wme_id);
            printf ("\n Step 1: %s",obj_string_buffer);
            /* Use the object's printform to make a copy */
            a_wme_id = rul_make_instance (obj_string_buffer, "");
            /* Modify the ^called attribute of the "made" copy */
            rul_set_attr_string (a_wme_id, "CALLED", "Neighbor");
            /* Print the read form of the "made" copy */
            b_len = rul_get_instance (obj_string_buffer, BUFF_SIZE, a_wme_id);
            printf ("\n Step 2: %s",obj_string_buffer);
            /* Copy the object made above... */
            new_wme_id = rul_copy_instance (a_wme_id);
            /* Modify the ^called attribute of the copy */
            rul_set_attr_string (new_wme_id, "CALLED", "TROUBLE");
            /* Print the read form of the "copied" copy */
            b_len = rul_get_instance (obj_string_buffer, BUFF_SIZE, new_wme_id);
            printf ("\n Step 3: %s",obj_string_buffer);
            /* remove the "made" copy */
            rul_atom_to_string (id_string_buffer, ID_BUFF_SIZE, a_wme_id);
            if (rul_remove_instance (a_wme_id))
                printf ("\n Removed Instance: %s\n", id_string_buffer);
            else
                printf ("\n Removal FAILED\n");
        }
        else
            printf ("\n INSTANCE-ID not a valid OBJECT\n");
    else
        printf ("\n Atom not an INSTANCE-ID\n");
    fflush (stdout);
}
\end{verbatim}
\end{quote}
\caption{Changing Working Memory: C Routine}
\label{e:6-9}
\end{example}

\begin{table}[h]
  \begin{tabularx}{\columnwidth}{lX}
    \toprule
    RTL Routine & Description \\
    \midrule
    \verb|rul_copy_instance| & Creates a new object with the same contents   
                               as an existing object.                        \\
    \verb|rul_end_id_translation| & Signals the end of an \verb|INSTANCE-ID|
                                    translation table. \\
    \verb|rul_make_instance| & Creates a new object from a string.           \\
    \verb|rul_remove_instance| & Deletes an object from working memory.        \\
    \verb|rul_set_attr_atom| & Changes the value of a scalar attribute to an 
                               atom.                                         \\
    \verb|rul_set_attr_double| & Changes the value of a scalar attribute to a  
                                 double float.                                 \\
    \verb|rul_set_attr_float| & Changes the value of a scalar attribute to a  
                                single float.                                 \\
    \verb|rul_set_attr_integer| & Changes the value of a scalar attribute to an 
                                  integer.                                      \\
    \verb|rul_set_attr_string| & Changes the value of a scalar attribute to a  
                                 string.                                       \\
    \verb|rul_set_comp_attr_string| & Changes the value of an entire compound       
                                      attribute to the values extracted from a      
                                      single string.                                \\
    \verb|rul_set_comp_elem_atom| & Changes the value of a single element of a    
                                    compound attribute to an atom.                \\
    \verb|rul_set_comp_elem_double| & Changes the value of a single element of a    
                                      compound attribute to a double-precision      
                                      floating-point number.                        \\
    \verb|rul_set_comp_elem_float|   & Changes the value of a single element of a    
                                       compound attribute to a single-precision      
                                       floating-point number.                        \\
    \verb|rul_set_comp_elem_integer| & Changes the value of a single element of a    
                                       compound attribute to an integer.             \\
    \verb|rul_set_comp_elem_string|  & Changes the value of a single element of a    
                                       compound attribute to a string.               \\
    \verb|rul_specialize_instance|   & Changes an instance of a parent class to an   
                                       instance of a subclass.                       \\
    \verb|rul_start_id_translation|  & Signals the creation of an INSTANCE-ID        
                                       translation table.                            \\
    \bottomrule
  \end{tabularx}
  \caption{RTL Routines for Changing Working Memory}
  \label{t:6-6}
\end{table}

Note: It may be necessary to call the appropriate declaration block
before calling RTL routines that test declarations or that create
WMOs, to ensure that the object classes have been initialized.

\begin{table}[h]
  \begin{tabularx}{\columnwidth}{lX}
    \toprule
    RTL Routine & Description \\
    \midrule
    \verb|rul_attr_is_compound|  & Indicates whether an attribute is compound or
                                   scalar. \\
    \verb|rul_is_attribute| & Indicates whether an attribute is declared
                              in the
                              specified object class. \\
    \verb|rul_is_class| & Indicates whether an object class with the
                          specified name has been declared. \\
    \verb|rul_is_subclass| & Indicates whether one object class inherits
                             from another. \\
    \bottomrule
  \end{tabularx}  
  \caption{RTL Routines for Testing Declarations}
  \label{t:6-7}
\end{table}

\begin{table}[h]
  \begin{tabularx}{\columnwidth}{lX}
    \toprule
    RTL Routine &  Description \\
    \midrule
    \verb|rul_atom_is_compound| & Indicates whether a value is compound or 
                                  scalar. \\
    \verb|rul_atom_is_fatom| & Indicates whether a value is a \tt{FLOAT} atom. \\
    \verb|rul_atom_is_iatom| & Indicates whether a value is an \tt{INTEGER} atom. \\
    \verb|rul_atom_is_instance_id| & Indicates whether a value is an \tt{INSTANCE-ID}
                                     atom. \\
    \verb|rul_atom_is_symbol| & Indicates whether a value is a \tt{SYMBOL} atom. \\
    \verb|rul_is_instance| & Indicates whether an object that corresponds to
                             the specified \tt{INSTANCE-ID} exists in working    
                             memory. \\
    \bottomrule
  \end{tabularx}
  \caption{RTL Routines for Testing Values}
  \label{t:6-8}
\end{table}

Example~\ref{e:6-11} is a C routine that displays the read form, print
form, and type of arbitrary atoms created by the RuleWorks program in
Example~\ref{e:6-10}.


Example 6-10 Testing and Converting Values: RuleWorks Program
\begin{quote}
\begin{verbatim}
(entry-block main)

(object-class foo ^bar)

(external-routine which_type_is_this (accepts atom))

(on-entry
    (make foo ^bar 1234)
    (make foo ^bar 43.21)
    (make foo ^bar |a symbol|)
    (run))

(rule any-atom
    (foo ^bar <x>)
  -->
    (which_type_is_this <x>))

(rule instance-id-atom
    (foo ^$id <x> ^bar 1234)
  -->
    (which_type_is_this <x>))

(end-block main)
\end{verbatim}
\end{quote}

Example 6-11 Testing and Converting Values: C Routine
\begin{quote}
\begin{verbatim}
#include <stdio.h>
#include <rul_rtl.h>

void WHICH_TYPE_IS_THIS (rul_atom atom_value)
{
    char tmp[RUL_C_MAX_SYMBOL_SIZE+1];
    long len;

    /* Get the read form of the given atom */
    rul_atom_to_string (tmp, RUL_C_MAX_SYMBOL_SIZE+1, atom_value);
    printf ("\n\n Atom has read form = '%s'", tmp);

    /* Print out type and value */
    if (rul_atom_is_iatom(atom_value)) {
        printf ("\n Atom is of type INTEGER");
        printf ("\n Atom has value = %12d",
             rul_iatom_to_integer (atom_value));
    }
    else if (rul_atom_is_fatom(atom_value)) {
        printf ("\n Atom is of type FLOAT");
        printf ("\n Atom has value = %12.4f", 
                rul_fatom_to_float (atom_value));
    }
    else if (rul_atom_is_symbol(atom_value)) {
        printf ("\n Atom is of type SYMBOL");
        len = rul_symbol_to_string (tmp, RUL_C_MAX_SYMBOL_SIZE+1,
                                    atom_value) ;
        printf ("\n Atom has print form = '%s'", tmp);
    }
    else if (rul_atom_is_instance_id(atom_value)) {
        printf ("\n Atom is of type INSTANCE_ID");
    }
    else {
        printf ("\n Atom is of unknown type");
    }
    fflush (stdout);
}
\end{verbatim}
\end{quote}

Running Example~\ref{e:6-10} produces the following output:

\begin{quote}
\begin{verbatim}
Atom has read form = '|a symbol|'
Atom is of type SYMBOL
Atom has print form = 'a symbol'

Atom has read form = '43.21'
Atom is of type FLOAT
Atom has value = 43.2100

Atom has read form = '#1'
Atom is of type INSTANCE_ID

Atom has read form = '1234'
Atom is of type INTEGER
Atom has value = 1234
\end{verbatim}
\end{quote}

\begin{table}[h]
  \begin{tabularx}{\columnwidth}{lX}
    \toprule
    RTL Routine &  Description \\
    \midrule
    \verb|rul_atom_to_string| & Converts an atom to a string. \\
    \verb|rul_atom_to_string_length| & Returns the number of characters in
                                       the string representation of an atom. \\
    \verb|rul_double_to_fatom| & Converts a double-precision floating-point 
                                 number into a FLOAT atom. \\
    \verb|rul_fatom_to_double| & Converts a FLOAT atom into a double-precision 
                                 floating-point number. \\
    \verb|rul_fatom_to_float| & Converts a FLOAT atom into a single-precision 
                                floating-point number.  \\
    \verb|rul_float_to_fatom| &  Converts a single-precision floating-point  
                                number into a FLOAT atom. \\
    \verb|rul_genint| & Generates a new INTEGER atom. \\
    \verb|rul_gensym| & Generates a new SYMBOL atom with the prefix \verb|G:.| \\    
    \verb|rul_gensymp| & Generates a new SYMBOL atom, with an optional prefix. \\    
    \verb|rul_iatom_to_integer| & Converts an INTEGER atom into an integer. \\
    \verb|rul_integer_to_iatom| & Converts an integer into an INTEGER atom. \\
    \verb|rul_string_to_atom| & Converts the first token of a string into an  
                                atom. \\
    \verb|rul_string_to_symbol| & Converts a character string into a SYMBOL atom. \\   
    \verb|rul_symbol_to_string| & Converts a SYMBOL atom into a character string. \\   
    \bottomrule
  \end{tabularx}
  \caption{RTL Routines for Converting Values}
  \label{t:6-9}
\end{table}

\begin{table}[h]
  \begin{tabularx}{\columnwidth}{lXXX}
    \toprule
    RTL Routine & Description \\
    \midrule
    \verb|rul_debug| & Invokes the RuleWorks command interpreter. \\
    \verb|rul_get_firing_rule| & Identifies the rule that the RuleWorks run-time 
                                 system is currently executing. \\
    \bottomrule
  \end{tabularx}
  \caption{RTL Routines for Controlling RuleWorks Execution}
  \label{t:6-10}
\end{table}

\subsubsection{Strings, Read Forms, and Print Forms}

The RTL routines listed in the first column of Table~\ref{t:6-11}
parse strings passed to them using the same semantics as the RuleWorks
reader. That is, their input should be a read form not a print
form. In general, any RTL routine that accepts or returns more than
one atom in a string uses read forms. RTL routines that use strings to
pass a single atom use a print form if the type of the atom is known;
if the type of the atom is not known the routines use a read form.

\begin{table}
  \centering
  \begin{tabular}{ll}
    \toprule
    Accept  & Return \\
    \midrule
    \verb|rul_make_instance| & \verb|rul_get_instance| \\
    \verb|rul_set_comp_attr_string| & \verb|rul_get_comp_attr_string| \\
    \verb|rul_string_to_atom| & \verb|rul_atom_to_string| \\
    \bottomrule
  \end{tabular}
  \caption{RTL Routines That Accept or Return Read Forms}
  \label{t:6-11}
\end{table}

Because the print form of a symbol can be different from its read
form, passing a print form to a routine that expects a read form can
cause unexpected results. The following C code creates two different
RuleWorks atoms, one whose print form is \tt{abc} and one whose print
form is \tt{ABC}. That is, \verb|my_atom| is not equal to
\verb|an_atom|.

\begin{quote}
\begin{verbatim}
#include <rul_rtl.h>
...
rul_atom my_atom, an_atom;
long len;
char buffer[RUL_C_MAX_SYMBOL_SIZE+1];
my_atom = rul_string_to_atom ("|abc|");
len = rul_symbol_to_string (&buffer, RUL_C_MAX_SYMBOL_SIZE+1, my_atom);
an_atom = rul_string_to_atom (&buffer);
\end{verbatim}
\end{quote}

The RTL routines listed in the second column of Table~\ref{t:6-11}
return read forms, not print forms. This allows your external routine
to use the string representations of RuleWorks objects.

\section{Handling an Interrupt}

RuleWorks programs can get information from external sources, such as
timers and I/O devices, by calling system routines that let the
programs request that they be interrupted when particular events
occur. An interrupt is called an asynchronous system trap (AST) on VMS
systems and a signal on UNIX systems. The system routine provides a
transfer of control to a user-specified procedure that handles the
event.

When a program calls a system routine, it typically specifies the
event handler as one of the arguments. The calling program then
continues to run until an event of the appropriate type occurs. When
the event occurs, the operating system interrupts the calling program
by immediately passing control to the event handler. When the event
handler finishes, the program continues from the point where it was
interrupted.

Normally, the event handler examines the event received, possibly
updates the program's data, and then returns control to the program at
the point the interruption occurred.

\textbf{CAUTION:} In a RuleWorks program the data (working memory) can
be updated at any point in the recognize-act cycle, but not from
interrupt level.  Therefore, interrupts have to be ``synchronized.''
An event handler cannot call RuleWorks to alter working
memory. Instead the event handler should modify some external
reentrant data structure. Another external routine that protects
itself from interrupts and knows how to poll that external data
structure should be called periodically, for example, from inside an
\tt{ON-EVERY} construct.

In summary, you must take the following steps for your RuleWorks
program to communicate with asynchronous or interrupt level external
sources:

\begin{itemize}
\item Create an external routine, called the polling routine, that
  passes information from the external reentrant data structure to the
  program by creating objects.
\item Call the polling routine periodically from your RuleWorks
  program.
\item Create another external routine, called the event handler, to
  receive the interrupts. This event handler routine executes at
  interrupt level and adds information to the reentrant external data
  structure.
\item Register the event handler routine with the operating system by
  calling the appropriate system routine.
\end{itemize}
             
\section{Summary of Restrictions}

This section lists the restrictions on using RuleWorks with other
languages.

On the Left-Hand Side:

\begin{itemize}
\item Do not call functions with hidden state, even ones as simple as
  fetching the value of an environment variable or logical name, or
  returning the number of times the function has been called. All
  functions used on the LHS should return a value whose computation
  depends directly and exclusively on the arguments passed.
\item Do not call any function that uses the RuleWorks RTL routines to
  read or change working memory in any way. These routines all depend
  on hidden state.
\end{itemize}

Anywhere in a Rule:

\begin{itemize}
\item Do not depend on the order of evaluation of argument expressions
  to any built-in action or function or to any external routine.
\item External routines that return allocated memory are responsible
  for deallocation of that memory (array or string, ASCIZ or ASCID).
\item Strings and arrays passed as return values from entry blocks are
  never deallocated, so when possible use read-write arguments
  instead.
\end{itemize}

In General:

\begin{itemize}             
\item The caller of any RuleWorks RTL routine is responsible for the
  allocation and deallocation of any memory required for the arguments
  to or from that routine.
\item Never call any RuleWorks RTL routine from interrupt level.
\end{itemize}

%%% Local Variables:
%%% mode: latex
%%% TeX-master: "rwug"
%%% End:
